\section*{social networks and bibliography}
\addcontentsline{toc}{section}{social networks and bibliography}

% $$$$$$$$$$$$$$$$$$$$$$$$$$$$$$$$$$$$$$$$$$$$$$$$$$$$$$$$$$$$$$$$$$$$$$$$$$$$$$$$

\subsection*{ricerca web}
\addcontentsline{toc}{subsection}{ricerca web}
\begin{flushleft}
	\begin{itemize}
		\item problematiche tipiche della ricerca web:
		\begin{enumerate}
			\item dinamicit\'a della collezione di pagine
			\item eterogeneit\'a dei contenuti
			\item abbondanza dei contenuti e delle pagine
			\item ridondanze e duplicazioni
			\item malvagit\'a degli autori
			\item ...
			\item hyperlinks
		\end{enumerate}
		\item \textbf{ricerca web:} information retrieval + spectral analysis + ??? (secret)
		\begin{itemize}
			\item information retrieval: tecniche di ricerca testuale
			\item spectral analysis: determinazione della rilevanza di una pagina come funzione della struttura a rete delle pagine web
			\item ??? (secret): segreto industriale + frequenti modifiche strategiche
		\end{itemize}
		\item gli hyperlinks forniscono informazione supplementare al normale testo e aiutano a definire delle misure di autorit\'a e popolarit\'a delle pagine (da annettere alle classiche misure di rilevanza)
		\item tali misure hanno avuto orgine delle reti sociali (soggetto di consistente ricerca prima dell'avvendo del web, il quale \'e una rete sociale)
		\item la teoria delle reti sociali e\'nteressante per la determinazione di propriet\'a correlate alla connetivit\'a e alla distanza nei grafi
		\begin{itemize}
			\item la centralit\'a
			\item la co-citazione
			\item il prestigio sociale
		\end{itemize}
	\end{itemize}
\end{flushleft}

% $$$$$$$$$$$$$$$$$$$$$$$$$$$$$$$$$$$$$$$$$$$$$$$$$$$$$$$$$$$$$$$$$$$$$$$$$$$$$$$$

\subsection*{centralit\'a}
\addcontentsline{toc}{subsection}{centralit\'a}
\begin{flushleft}
	\begin{itemize}
		\item molte nozioni di centralit\'a basate sui grafi sono state proposte nella letteratura delle reti sociali
		\item tuttavia, tra le innumerevoli formulazioni e misure, nessuna si adatta perfettamente a tutti gli scenari
		\item $d(u,v)=$ distanza tra 2 nodi $u$ e $v$ in un grafo
		\begin{enumerate}
			\item non pesato: uguale al numero di archi del cammino minimo tra $u$ e $v$
			\item pesato: uguale al costo degli archi del cammino minimo tra $u$ e $v$
		\end{enumerate}
		\item $r(u)=\max_vd(u,v)$ raggio di un nodo $u$
	\end{itemize}
\end{flushleft}

% $$$$$$$$$$$$$$$$$$$$$$$$$$$$$$$$$$$$$$$$$$$$$$$$$$$$$$$$$$$$$$$$$$$$$$$$$$$$$$$$

\subsection*{alcune misure di centralit\'a}
\addcontentsline{toc}{subsection}{alcune misure di centralit\'a}
\begin{flushleft}
	\begin{itemize}
		\item \textbf{centralit\'a di raggio (radius)}:
		\begin{itemize}
			\item[] $$c[v]=\min r(v)$$
		\end{itemize}
		\item \textbf{centralit\'a di grado (degree)}:
		\begin{itemize}
			\item[] $$c[v]=\max deg(v)$$
		\end{itemize}
		\item \textbf{centralit\'a di vicinanza (closeness)}:
		\begin{itemize}
			\item[] $$c[v]=\max\sum_{t\in V,t\neq v}\frac{1}{d(v,t)}$$
		\end{itemize}
		\item \textbf{centralit\'a di interazione (betweenness)}:
		\begin{itemize}
			\item[] $$c[v]=\max\sum_{s,t\in V,s\neq t\neq v}\frac{\sigma_{s,t}(v)}{\sigma_{s,t}}$$
		\end{itemize}
	\end{itemize}
\end{flushleft}

% $$$$$$$$$$$$$$$$$$$$$$$$$$$$$$$$$$$$$$$$$$$$$$$$$$$$$$$$$$$$$$$$$$$$$$$$$$$$$$$$

\subsection*{}
\addcontentsline{toc}{subsection}{}
\begin{flushleft}
	\begin{itemize}
		\item aaa
	\end{itemize}
\end{flushleft}

% $$$$$$$$$$$$$$$$$$$$$$$$$$$$$$$$$$$$$$$$$$$$$$$$$$$$$$$$$$$$$$$$$$$$$$$$$$$$$$$$

\subsection*{}
\addcontentsline{toc}{subsection}{}
\begin{flushleft}
	\begin{itemize}
		\item aaa
	\end{itemize}
\end{flushleft}

% $$$$$$$$$$$$$$$$$$$$$$$$$$$$$$$$$$$$$$$$$$$$$$$$$$$$$$$$$$$$$$$$$$$$$$$$$$$$$$$$

\subsection*{}
\addcontentsline{toc}{subsection}{}
\begin{flushleft}
	\begin{itemize}
		\item aaa
	\end{itemize}
\end{flushleft}

% $$$$$$$$$$$$$$$$$$$$$$$$$$$$$$$$$$$$$$$$$$$$$$$$$$$$$$$$$$$$$$$$$$$$$$$$$$$$$$$$

\subsection*{}
\addcontentsline{toc}{subsection}{}
\begin{flushleft}
	\begin{itemize}
		\item aaa
	\end{itemize}
\end{flushleft}

% $$$$$$$$$$$$$$$$$$$$$$$$$$$$$$$$$$$$$$$$$$$$$$$$$$$$$$$$$$$$$$$$$$$$$$$$$$$$$$$$

\subsection*{}
\addcontentsline{toc}{subsection}{}
\begin{flushleft}
	\begin{itemize}
		\item aaa
	\end{itemize}
\end{flushleft}

% $$$$$$$$$$$$$$$$$$$$$$$$$$$$$$$$$$$$$$$$$$$$$$$$$$$$$$$$$$$$$$$$$$$$$$$$$$$$$$$$

\newpage
