\section*{optimization problems}
\addcontentsline{toc}{section}{optimization problems}

% $$$$$$$$$$$$$$$$$$$$$$$$$$$$$$$$$$$$$$$$$$$$$$$$$$$$$$$$$$$$$$$$$$$$$$$$$$$$$$$$

\subsection*{def: problema di ottimizzazione}
\addcontentsline{toc}{subsection}{def: problema di ottimizzazione}
\begin{flushleft}
	un problema di ottimizzazione $\pi$ \'e una quadrupla $(I_{\pi}, S_{\pi}, m_{\pi}, goal_{\pi})$ con:
	\begin{itemize}
		\item $I_{\pi}=$ insieme delle istanze di input di $\pi$
		\item $S_{\pi}(x)=$ insieme delle soluzioni ammissibili dell'istanza $x\in I_{\pi}$
		\item $m_{\pi}(x,y)=$ misura della soluzione ammissibile $y\in S_{\pi}(x)$ per l'input $x\in I_{\pi}$ (intera)
		\item $goal_{\pi}\in\{\min,\max\}=$ specifica se abbiamo un problema di minimizzazione o di massimizzazione
	\end{itemize}
\end{flushleft}

% $$$$$$$$$$$$$$$$$$$$$$$$$$$$$$$$$$$$$$$$$$$$$$$$$$$$$$$$$$$$$$$$$$$$$$$$$$$$$$$$

\subsection*{osservazioni (problemi di ottimizzazione)}
\addcontentsline{toc}{subsection}{osservazioni (problemi di ottimizzazione)}
\begin{flushleft}
	\begin{itemize}
		\item assumiamo che $m_{\pi}(x,y)$ \'e sempre un numero intero
		\begin{itemize}
			\item i nostri modelli computazionali possono trattare solo l'approssimazione razionale dei reali
			\item scalando tali reali possiamo ottenere numeri interi equivalenti
			\item i valori interi rivelano gi\'a le difficolt\'a intrinseche dei problemi
		\end{itemize}
		\item quando sono chiari dal contesto (in seguito):
		\begin{itemize}
			\item $\pi$ sar\'a omesso
			\item $m_(x,y)=$ sar\'a denotato semplicemente come $m$ 
		\end{itemize}
	\end{itemize}
\end{flushleft}

% $$$$$$$$$$$$$$$$$$$$$$$$$$$$$$$$$$$$$$$$$$$$$$$$$$$$$$$$$$$$$$$$$$$$$$$$$$$$$$$$

\subsection*{esempio: descrizione formale di un problema di ottimizzazione (max clique)}
\addcontentsline{toc}{subsection}{esempio: descrizione formale di un problema di ottimizzazione (max clique)}
\begin{flushleft}
	\begin{itemize}
		\item $I=$ grafo $G=(V,E)$
		\item $S= \{U\subseteq V\hspace{0.1cm}\vert\hspace{0.1cm}\{u,v\}\in E,\hspace{0.1cm}\forall u,v\in U\}$
		\item $m(G,U)=|U|$
		\item $goal=\max$
	\end{itemize}
	\vspace{0.5cm}
	possiamo descrivere i problemi di ottimizzazione nella seguente forma, pi\'u semplice e informale
	\begin{itemize}
		\item MAX CLIQUE
		\begin{itemize}
			\item INPUT: grafo $G=(V,E)$
			\item SOLUZIONE: $U\subseteq V\hspace{0.1cm}\vert\hspace{0.1cm}\{u,v\}\in E,\hspace{0.1cm}\forall u,v\in U$
			\item MISURA: $|U|$
		\end{itemize}
	\end{itemize}
\end{flushleft}

% $$$$$$$$$$$$$$$$$$$$$$$$$$$$$$$$$$$$$$$$$$$$$$$$$$$$$$$$$$$$$$$$$$$$$$$$$$$$$$$$

\subsection*{def: soluzione ottima}
\addcontentsline{toc}{subsection}{def: soluzione ottima}
\begin{flushleft}
	\begin{itemize}
		\item data un'istanza $x\in I_\pi$, una soluzione $y^*\in S_\pi$ \'e ottima per $x$ se $m(x,y^*)=goal\{m(x,y)\hspace{0.1cm}\vert\hspace{0.1cm}y\in S(x)\}$
	\item la misura di una soluzione ottima (o in modo analogo di tutte le soluzioni ottime) di $x$ \'e denotata come $m^*(x)$ o semplicemente $m^*$
	\end{itemize}
\end{flushleft}

% $$$$$$$$$$$$$$$$$$$$$$$$$$$$$$$$$$$$$$$$$$$$$$$$$$$$$$$$$$$$$$$$$$$$$$$$$$$$$$$$

\subsection*{problema decisionale sottostante}
\addcontentsline{toc}{subsection}{problema decisionale sottostante}
\begin{flushleft}
	ogni problema di ottimizzazione ha un problema decisionale sottostante che pu\'o essere ottenuto introducendo un intero $k$ nell'istanza di input e chiedendo se esiste una soluzione ammissibile di misura $\leq k$ (per $\min$) e $\geq k$ (per $\max$)
	\begin{itemize}
		\item problema di ottimizzazione:
		\begin{itemize}
			\item dato un input $x$, trova $y\in S(x)\hspace{0.1cm}\vert\hspace{0.1cm}m(x,y)$ sia $\min$ o $\max$ (secondo il $goal$)
		\end{itemize}
		\item problema decisionale sottostante:
		\begin{itemize}
			\item dato un input $x$ e un intero $k\geq 0$, esiste $y\in S(x)\hspace{0.1cm}\vert\hspace{0.1cm}m(x,y)\leq k$ ($\min$) o $\geq k$ ($\max$)
		\end{itemize}
	\end{itemize}
\end{flushleft}

% $$$$$$$$$$$$$$$$$$$$$$$$$$$$$$$$$$$$$$$$$$$$$$$$$$$$$$$$$$$$$$$$$$$$$$$$$$$$$$$$

\subsection*{esempio: descrizione formale di un problema decisionale sottostante (max clique)}
\addcontentsline{toc}{subsection}{esempio: descrizione formale di un problema decisionale sottostante (max clique)}
\begin{flushleft}
	\begin{itemize}
		\item MAX CLIQUE
		\begin{itemize}
			\item INPUT: grafo $G=(V,E)$
			\item SOLUZIONE: $U\subseteq V\hspace{0.1cm}\vert\hspace{0.1cm}\{u,v\}\in E,\hspace{0.1cm}\forall u,v\in U$
			\item MISURA: $|U|$
		\end{itemize}
		\item problema decisionale sottostante:
		\begin{itemize}
			\item INPUT: grafo $G=(V,E)$ e un intero $k>0$
			\item DOMANDA: esiste una clique $U$ in $G$ tale che $|U|\geq k$
		\end{itemize}
	\end{itemize}
\end{flushleft}

% $$$$$$$$$$$$$$$$$$$$$$$$$$$$$$$$$$$$$$$$$$$$$$$$$$$$$$$$$$$$$$$$$$$$$$$$$$$$$$$$

\subsection*{}
\addcontentsline{toc}{subsection}{}
\begin{flushleft}
\end{flushleft}

% $$$$$$$$$$$$$$$$$$$$$$$$$$$$$$$$$$$$$$$$$$$$$$$$$$$$$$$$$$$$$$$$$$$$$$$$$$$$$$$$

\subsection*{}
\addcontentsline{toc}{subsection}{}
\begin{flushleft}
\end{flushleft}

% $$$$$$$$$$$$$$$$$$$$$$$$$$$$$$$$$$$$$$$$$$$$$$$$$$$$$$$$$$$$$$$$$$$$$$$$$$$$$$$$

\subsection*{}
\addcontentsline{toc}{subsection}{}
\begin{flushleft}
\end{flushleft}

% $$$$$$$$$$$$$$$$$$$$$$$$$$$$$$$$$$$$$$$$$$$$$$$$$$$$$$$$$$$$$$$$$$$$$$$$$$$$$$$$

\subsection*{}
\addcontentsline{toc}{subsection}{}
\begin{flushleft}
\end{flushleft}

% $$$$$$$$$$$$$$$$$$$$$$$$$$$$$$$$$$$$$$$$$$$$$$$$$$$$$$$$$$$$$$$$$$$$$$$$$$$$$$$$

\subsection*{}
\addcontentsline{toc}{subsection}{}
\begin{flushleft}
\end{flushleft}

% $$$$$$$$$$$$$$$$$$$$$$$$$$$$$$$$$$$$$$$$$$$$$$$$$$$$$$$$$$$$$$$$$$$$$$$$$$$$$$$$

\subsection*{}
\addcontentsline{toc}{subsection}{}
\begin{flushleft}
\end{flushleft}

% $$$$$$$$$$$$$$$$$$$$$$$$$$$$$$$$$$$$$$$$$$$$$$$$$$$$$$$$$$$$$$$$$$$$$$$$$$$$$$$$

\subsection*{}
\addcontentsline{toc}{subsection}{}
\begin{flushleft}
\end{flushleft}

% $$$$$$$$$$$$$$$$$$$$$$$$$$$$$$$$$$$$$$$$$$$$$$$$$$$$$$$$$$$$$$$$$$$$$$$$$$$$$$$$

\subsection*{}
\addcontentsline{toc}{subsection}{}
\begin{flushleft}
\end{flushleft}

% $$$$$$$$$$$$$$$$$$$$$$$$$$$$$$$$$$$$$$$$$$$$$$$$$$$$$$$$$$$$$$$$$$$$$$$$$$$$$$$$

\subsection*{}
\addcontentsline{toc}{subsection}{}
\begin{flushleft}
\end{flushleft}

% $$$$$$$$$$$$$$$$$$$$$$$$$$$$$$$$$$$$$$$$$$$$$$$$$$$$$$$$$$$$$$$$$$$$$$$$$$$$$$$$

\subsection*{}
\addcontentsline{toc}{subsection}{}
\begin{flushleft}
\end{flushleft}

% $$$$$$$$$$$$$$$$$$$$$$$$$$$$$$$$$$$$$$$$$$$$$$$$$$$$$$$$$$$$$$$$$$$$$$$$$$$$$$$$

\subsection*{}
\addcontentsline{toc}{subsection}{}
\begin{flushleft}
\end{flushleft}

% $$$$$$$$$$$$$$$$$$$$$$$$$$$$$$$$$$$$$$$$$$$$$$$$$$$$$$$$$$$$$$$$$$$$$$$$$$$$$$$$

\subsection*{}
\addcontentsline{toc}{subsection}{}
\begin{flushleft}
\end{flushleft}

% $$$$$$$$$$$$$$$$$$$$$$$$$$$$$$$$$$$$$$$$$$$$$$$$$$$$$$$$$$$$$$$$$$$$$$$$$$$$$$$$

\subsection*{}
\addcontentsline{toc}{subsection}{}
\begin{flushleft}
\end{flushleft}

\newpage
