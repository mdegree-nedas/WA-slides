\section*{alternative approaches}
\addcontentsline{toc}{section}{alternative approaches}

% $$$$$$$$$$$$$$$$$$$$$$$$$$$$$$$$$$$$$$$$$$$$$$$$$$$$$$$$$$$$$$$$$$$$$$$$$$$$$$$$

\subsection*{performance garantite}
\addcontentsline{toc}{subsection}{performance garantite}
\begin{flushleft}
	\begin{itemize}
		\item finora abbiamo considerato approcci con performance garantite
		\item pro:
		\begin{itemize}
			\item approssimazione e tempo di esecuzione garantiti per ogni istanza in input
			\item prende in considerazione il caso peggiore
		\end{itemize}
		\item contro:
		\begin{itemize}
			\item alcuni problemi non ammettono algoritmi con performance garantite
			\item per alcuni problemi non sono noti algoritmi con performance garantite
			\item a volte, cattivi comportamenti nella pratica
		\end{itemize}
	\end{itemize}
\end{flushleft}

% $$$$$$$$$$$$$$$$$$$$$$$$$$$$$$$$$$$$$$$$$$$$$$$$$$$$$$$$$$$$$$$$$$$$$$$$$$$$$$$$

\subsection*{restrizione dell'insieme delle istanze}
\addcontentsline{toc}{subsection}{restrizione dell'insieme delle istanze}
\begin{flushleft}
	\begin{itemize}
		\item performance garantite nel sottoinsieme delle istanze in input che sono significative o di interesse
		\item pro:
		\begin{itemize}
			\item permette di applicare nuovamente l'approccio con performance garantite
		\end{itemize}
		\item contro:
		\begin{itemize}
			\item performance garantite solo per il sottoinsieme scelto di istanze o per un caso particolare
		\end{itemize}
		\item esempio: metric TSP (con disequazioni triangolari)
	\end{itemize}
\end{flushleft}

% $$$$$$$$$$$$$$$$$$$$$$$$$$$$$$$$$$$$$$$$$$$$$$$$$$$$$$$$$$$$$$$$$$$$$$$$$$$$$$$$

\subsection*{media o analisi probabilistica}
\addcontentsline{toc}{subsection}{media o analisi probabilistica}
\begin{flushleft}
	\begin{itemize}
		\item in generale, assumendo una distribuzione di probabilit\'a dell'istanza, essa eguaglia la media o la performance attesa, alcune volte con alta probabilit\'a
		\item pro:
		\begin{itemize}
			\item pu\'o accorgersi di buoni comportamenti pratici dell'algoritmo
			\item \'e un metodo analitico, ovvero basato su dimostrazioni matematiche
		\end{itemize}
		\item contro:
		\begin{itemize}
			\item non ha performance garantite
			\item l'analisi \'e spesso complessa
			\item spesso la distribuzione delle istanze in input \'e sconosciuta
		\end{itemize}
	\end{itemize}
\end{flushleft}

% $$$$$$$$$$$$$$$$$$$$$$$$$$$$$$$$$$$$$$$$$$$$$$$$$$$$$$$$$$$$$$$$$$$$$$$$$$$$$$$$

\subsection*{euristiche}
\addcontentsline{toc}{subsection}{euristiche}
\begin{flushleft}
	\begin{itemize}
		\item algoritmi con un buon comportamento pratico ma solitamente ocn performance non dimostrabili
		\item pro:
		\begin{itemize}
			\item buon comportamento pratico
		\end{itemize}
		\item contro:
		\begin{itemize}
			\item performance spesso non dimostrabili
		\end{itemize}
	\end{itemize}
\end{flushleft}

% $$$$$$$$$$$$$$$$$$$$$$$$$$$$$$$$$$$$$$$$$$$$$$$$$$$$$$$$$$$$$$$$$$$$$$$$$$$$$$$$

\subsection*{algoritmi randomizzati}
\addcontentsline{toc}{subsection}{algoritmi randomizzati}
\begin{flushleft}
	\begin{itemize}
		\item effetuano scelte randomiche durante la computazione
		\item le soluzioni ritornate possono essere differenti per esecuzioni differenti sullo stesso input
		\item esse sono infatti variabili random (per ogni istanza vi sono diverse soluzioni, ciascuna restituita con una certa probabilit\'a determinata in accordo con le scelte randomiche dell'algoritmo)
		\item \'e mostrato che, fissata una qualsiasi istanza, il valore atteso delle performance \'e buono o la performance \'e buona con alta probabilit\'a (sempre in accordo con le scelte randomiche)
		\item pro:
		\begin{itemize}
			\item sono generalmente semplici
			\item sono veloci (sia analiticamente che in pratica)
		\end{itemize}
		\item contro:
		\begin{itemize}
			\item incertezza del risultato per ogni istanza fissata
			\item impossibilit\'a di fare scelte realmente randomiche (sebbene esse possano essere simulate)
		\end{itemize}
	\end{itemize}
\end{flushleft}

% $$$$$$$$$$$$$$$$$$$$$$$$$$$$$$$$$$$$$$$$$$$$$$$$$$$$$$$$$$$$$$$$$$$$$$$$$$$$$$$$

\subsection*{algoritmi randomizzati}
\addcontentsline{toc}{subsection}{algoritmi randomizzati}
\begin{flushleft}
	\begin{itemize}
		\item $m\rightarrow$ \'e una variabile randomica
		\item $E(m)\rightarrow$ \'e il valore atteso di $m$ calcolato in accordo con le scelte randomiche dell'algoritmo
	\end{itemize}
\end{flushleft}

% $$$$$$$$$$$$$$$$$$$$$$$$$$$$$$$$$$$$$$$$$$$$$$$$$$$$$$$$$$$$$$$$$$$$$$$$$$$$$$$$

\subsection*{definizione: algoritmi randomizzati e $r$-approssimazione}
\addcontentsline{toc}{subsection}{definizione: algoritmi randomizzati e $r$-approssimazione}
\begin{flushleft}
	\begin{itemize}
		\item un algoritmo randomizzato $A$ \'e $r$-approssimante se:
			$$\frac{E(m)}{m^*}\leq r\hspace{2cm}\text{(per }\min\text{)}$$
			$$\frac{E(m)}{m^*}\geq r\hspace{2cm}\text{(per }\max\text{)}$$
	\end{itemize}
\end{flushleft}

% $$$$$$$$$$$$$$$$$$$$$$$$$$$$$$$$$$$$$$$$$$$$$$$$$$$$$$$$$$$$$$$$$$$$$$$$$$$$$$$$

\subsection*{problema: max weighted cut}
\addcontentsline{toc}{subsection}{problema: max weighted cut}
\begin{flushleft}
	\begin{itemize}
		\item INPUT:
		\begin{itemize}
			\item grafo $G=(V,E)$
			\item peso non-negativo $w_{ij}>0$, $\forall\{v_i,v_j\}\in E$
		\end{itemize}
		\item SOLUZIONE: partizione di $V$ in 2 sottoinsiemi $V_1$ e $V_2$ tale che
			$$V_1\cap V_2=\emptyset\text{ e }V_1\cup V_2=V$$
		\item MISURA: peso del taglio, ovvero
			$$\sum_{\{v_i,v_j\}\in E\vert v_i\in V_1\wedge v_j\in V_2}w_{ij}$$
	\end{itemize}
\end{flushleft}

% $$$$$$$$$$$$$$$$$$$$$$$$$$$$$$$$$$$$$$$$$$$$$$$$$$$$$$$$$$$$$$$$$$$$$$$$$$$$$$$$

\newpage
\subsection*{algoritmo: Random-Cut}
\addcontentsline{toc}{subsection}{algoritmo: Random-Cut}
\begin{flushleft}
	\begin{algorithm}
		\caption{Random-Cut}
		\begin{algorithmic}
			\STATE $V_1=\emptyset$
			\STATE $V_2=\emptyset$
			\FOR{$i=1$ to $n$}
				\STATE inserisci $v_i$ in $V_1$ con probabilit\'a $\frac{1}{2}$ indipendentemente dagli altri nodi (oppure in $V_2$)
			\ENDFOR
			\RETURN $V_1$ e $V\setminus V_1$ ($\equiv V_2$)
		\end{algorithmic}
	\end{algorithm}
	\begin{itemize}
		\item chiaramente, l'algoritmo \'e polinomale
	\end{itemize}
\end{flushleft}

% $$$$$$$$$$$$$$$$$$$$$$$$$$$$$$$$$$$$$$$$$$$$$$$$$$$$$$$$$$$$$$$$$$$$$$$$$$$$$$$$

\subsection*{}
\addcontentsline{toc}{subsection}{}
\begin{flushleft}
\end{flushleft}

% $$$$$$$$$$$$$$$$$$$$$$$$$$$$$$$$$$$$$$$$$$$$$$$$$$$$$$$$$$$$$$$$$$$$$$$$$$$$$$$$

\subsection*{}
\addcontentsline{toc}{subsection}{}
\begin{flushleft}
\end{flushleft}

% $$$$$$$$$$$$$$$$$$$$$$$$$$$$$$$$$$$$$$$$$$$$$$$$$$$$$$$$$$$$$$$$$$$$$$$$$$$$$$$$

\subsection*{}
\addcontentsline{toc}{subsection}{}
\begin{flushleft}
\end{flushleft}

% $$$$$$$$$$$$$$$$$$$$$$$$$$$$$$$$$$$$$$$$$$$$$$$$$$$$$$$$$$$$$$$$$$$$$$$$$$$$$$$$

\subsection*{}
\addcontentsline{toc}{subsection}{}
\begin{flushleft}
\end{flushleft}

% $$$$$$$$$$$$$$$$$$$$$$$$$$$$$$$$$$$$$$$$$$$$$$$$$$$$$$$$$$$$$$$$$$$$$$$$$$$$$$$$

\subsection*{}
\addcontentsline{toc}{subsection}{}
\begin{flushleft}
\end{flushleft}

% $$$$$$$$$$$$$$$$$$$$$$$$$$$$$$$$$$$$$$$$$$$$$$$$$$$$$$$$$$$$$$$$$$$$$$$$$$$$$$$$

\subsection*{}
\addcontentsline{toc}{subsection}{}
\begin{flushleft}
\end{flushleft}

% $$$$$$$$$$$$$$$$$$$$$$$$$$$$$$$$$$$$$$$$$$$$$$$$$$$$$$$$$$$$$$$$$$$$$$$$$$$$$$$$

\newpage
