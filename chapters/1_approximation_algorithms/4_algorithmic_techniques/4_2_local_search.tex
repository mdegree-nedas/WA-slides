\section*{algorithmic techniques: local search}
\addcontentsline{toc}{section}{algorithmic techniques: local search}

% $$$$$$$$$$$$$$$$$$$$$$$$$$$$$$$$$$$$$$$$$$$$$$$$$$$$$$$$$$$$$$$$$$$$$$$$$$$$$$$$

\subsection*{caratteristiche}
\addcontentsline{toc}{subsection}{caratteristiche}
\begin{flushleft}
	\begin{itemize}
		\item definiamo, per ogni soluzione ammissibile $y$, un sottoinsieme di soluzioni ammissibili "vicine" chiamato intorno di $y$ o semplicemente $neighborhood(y)$
		\item partendo da una soluzione iniziale, si passa ripetutamente ad una soluzione migliore nell'intorno corrente, finch\'e possibile
	\end{itemize}
\end{flushleft}

% $$$$$$$$$$$$$$$$$$$$$$$$$$$$$$$$$$$$$$$$$$$$$$$$$$$$$$$$$$$$$$$$$$$$$$$$$$$$$$$$

\subsection*{schema di un algoritmo di ricerca locale}
\addcontentsline{toc}{subsection}{schema di un algoritmo di ricerca locale}
\begin{flushleft}
	\begin{itemize}
		\item risolve una soluzione iniziale $y$ ammissibile per l'input $x$ (di solito una banale)
		\item fintanto che esiste una $y'\in neighborhood(y)$ migliore di $y$
		\begin{itemize}
			\item sia $y=y'$
		\end{itemize}
		\item ritorna $y$
		\item per definire un algoritmo di ricerca locale per un determinato problema \'e quindi sufficiente definire:
		\begin{itemize}
			\item la soluzione iniziale
			\item l'intorno delle soluzioni ammissibili
		\end{itemize}
	\end{itemize}
\end{flushleft}

% $$$$$$$$$$$$$$$$$$$$$$$$$$$$$$$$$$$$$$$$$$$$$$$$$$$$$$$$$$$$$$$$$$$$$$$$$$$$$$$$

\subsection*{complessit\'a}
\addcontentsline{toc}{subsection}{complessit\'a}
\begin{flushleft}
	\begin{itemize}
		\item per ottenere una complessit\'a temporale polinomiale:
		\begin{itemize}
			\item la soluzione iniziale deve essere determinata in tempo polinomiale
			\item il test della condizione di guardia del $while$ e l'eventuale conseguente determinazione di una soluzione migliore nell'intorno deve essere eseguito in tempo polinomiale
			\item NOTA: l'intorno pu\'o avere una cardinalit\'a esponenziale rispetto alla dimensione dell'input!
			\item il numero di iterazioni del $while$ deve essere polinomiale
		\end{itemize}
	\end{itemize}
\end{flushleft}

% $$$$$$$$$$$$$$$$$$$$$$$$$$$$$$$$$$$$$$$$$$$$$$$$$$$$$$$$$$$$$$$$$$$$$$$$$$$$$$$$

\subsection*{approssimazione}
\addcontentsline{toc}{subsection}{approssimazione}
\begin{flushleft}
	\begin{itemize}
		\item OTTIMO LOCALE: la soluzione $y$ restituita \'e la migliore nell'intorno considerato
		\item per limitare il rapporto di approssimazione \'e sufficiente limitare il rapporto tra il valore di un qualsiasi ottimo locale con quello della misura di una soluzione ottima globale
	\end{itemize}
\end{flushleft}

% $$$$$$$$$$$$$$$$$$$$$$$$$$$$$$$$$$$$$$$$$$$$$$$$$$$$$$$$$$$$$$$$$$$$$$$$$$$$$$$$

\subsection*{definizione dell'intorno}
\addcontentsline{toc}{subsection}{definizione dell'intorno}
\begin{flushleft}
	\begin{itemize}
		\item $neighborhood(y)$:
		\begin{itemize}
			\item sufficientemente "ricco", per ottenere buone soluzioni (ottimi locali)
			\item sufficientemente "povero", per garantire una complessit\'a temporale polinomiale
		\end{itemize}
	\end{itemize}
\end{flushleft}

% $$$$$$$$$$$$$$$$$$$$$$$$$$$$$$$$$$$$$$$$$$$$$$$$$$$$$$$$$$$$$$$$$$$$$$$$$$$$$$$$

\subsection*{definizione dell'intorno: casi estremi}
\addcontentsline{toc}{subsection}{definizione dell'intorno: casi estremi}
\begin{flushleft}
	\begin{itemize}
		\item $neighborhood(y)=\emptyset$
		\begin{itemize}
			\item tempo di esecuzione polinomiale (se la soluzione iniziale viene determinata in tempo polinomiale)
			\item cattiva approssimazione (ogni soluzione \'e un ottimo locale)
		\end{itemize}
		\item $neighborhood(y)=S(x)$, ovvero l'insieme di tutte le soluzioni ammissibili per $x$
		\begin{itemize}
			\item tempo di esecuzione non polinomiale (se il problema \'e NP-HARD)
			\item buona approssimazione (poich\'e ogni ottimo locale \'e anche un ottimo globale)
		\end{itemize}
	\end{itemize}
\end{flushleft}

% $$$$$$$$$$$$$$$$$$$$$$$$$$$$$$$$$$$$$$$$$$$$$$$$$$$$$$$$$$$$$$$$$$$$$$$$$$$$$$$$

\subsection*{problema: max cut (gi\'a definito precedentemente)}
\addcontentsline{toc}{subsection}{problema: max cut (gi\'a definito precedentemente)}
\begin{flushleft}
	\begin{itemize}
		\item INPUT: grafo $G=(V,E)$
		\item SOLUZIONE: una partizione di $V$ in 2 sottoinsiemi $V_1$ e $V_2$, ovvero tale che:
			$$V_1\cup V_2=V\text{ e }V_1\cap V_2=\emptyset$$
		\item MISURA: la cardinalit\'a del taglio, ovvero il numero di archi con un estremo (nodo) in $V_1$ e un estremo in $V_2$, cio\'e:
			$$\vert\{\{u,v\}\hspace{0.1cm}\vert\hspace{0.1cm}u\in V_1\text{ e }v\in V_2\}\vert$$
	\end{itemize}
\end{flushleft}

% $$$$$$$$$$$$$$$$$$$$$$$$$$$$$$$$$$$$$$$$$$$$$$$$$$$$$$$$$$$$$$$$$$$$$$$$$$$$$$$$

\subsection*{algoritmo di ricerca locale per max cut}
\addcontentsline{toc}{subsection}{algoritmo di ricerca locale per max cut}
\begin{flushleft}
	\begin{itemize}
		\item per definire l'algoritmo di ricerca locale, \'e sufficiente determinare:
		\begin{itemize}
			\item la soluzione iniziale:
				$$V_1=V\text{, }V_2=\emptyset$$
			\item l'intorno:
			\begin{itemize}
				\item dati $V=\{v_1,\ldots,v_n\}$ e $V_1$, $V_2$, le soluzioni dell'intorno di $(V_1,V_2)$ sono tutte le coppie $(V_{1i},V_{2i})$ con $1\leq i\leq n$ che possono essere ottenute muovendo un nodo $v_i$ da $V_1$ a $V_2$ o viceversa, ovvero:
					$$\text{if }(v_i\in V_1)\text{ }V_{1i}=V_1\setminus\{v_i\}\text{ e }V_{2i}=V_2\cup\{v_i\}$$
					$$\text{else }(v_i\in V_2)\text{ }V_{1i}=V_1\cup\{v_i\}\text{ e }V_{2i}=V_2\setminus\{v_i\}$$
			\end{itemize}
		\end{itemize}
	\end{itemize}
\end{flushleft}

% $$$$$$$$$$$$$$$$$$$$$$$$$$$$$$$$$$$$$$$$$$$$$$$$$$$$$$$$$$$$$$$$$$$$$$$$$$$$$$$$

\subsection*{complessit\'a (algoritmo di ricerca locale per max cut)}
\addcontentsline{toc}{subsection}{complessit\'a (algoritmo di ricerca locale per max cut)}
\begin{flushleft}
	\begin{itemize}
		\item la soluzione iniziale viene banalmente ottenuta in tempo polinomiale
		\item il test della guardia $while$ e l'eventuale determinazione di una migliore soluzione nell'intorno viene effettuata in tempo polinomale come segue:
		\begin{itemize}
			\item per ciascuna delle $n$ soluzioni dell'intorno ($n$ iterazioni), controlla se la soluzione corrente \'e migliore ($n^2$ iterazioni) $\rightarrow O(n^3)$
		\end{itemize}
		\item le iterazioni nel $while$ sono al massimo ($\leq$) $\vert E\vert=O(n^2)$,poich\'e ogni iterazione migliora la soluzione corrente, ovvero aumenta  almeno di $1$ il numero di archi del taglio, e vi sono $\vert E\vert$ archi nel taglio (al massimo)
		\item quindi l'algoritmo ha complessit\'a temporale:
			$$O(n^3n^2)=O(n^5)$$
	\end{itemize}
\end{flushleft}

% $$$$$$$$$$$$$$$$$$$$$$$$$$$$$$$$$$$$$$$$$$$$$$$$$$$$$$$$$$$$$$$$$$$$$$$$$$$$$$$$

\subsection*{approssimazione (algoritmo di ricerca locale per max cut)}
\addcontentsline{toc}{subsection}{approssimazione (algoritmo di ricerca locale per max cut)}
\begin{flushleft}
	\begin{itemize}
		\item vediamo una propriet\'a utile a mostrare il rapporto di approssimazione dell'algoritmo:
	\end{itemize}
\end{flushleft}

% $$$$$$$$$$$$$$$$$$$$$$$$$$$$$$$$$$$$$$$$$$$$$$$$$$$$$$$$$$$$$$$$$$$$$$$$$$$$$$$$

\subsection*{fatto (approssimazione (algoritmo di ricerca locale per max cut))}
\addcontentsline{toc}{subsection}{fatto (approssimazione (algoritmo di ricerca locale per max cut))}
\begin{flushleft}
	dato un grafo $G=(V,E)$, sia $\delta_i$ il grado di un generico nodo $v_i\in V$, allora:
	$$\sum_{i=1}^n\delta_i=2\vert E\vert$$
	\vspace{0.5cm}
	\textbf{dimostrazione:}
	\begin{itemize}
		\item banalmente vero, poich\'e ogni arco viene contato $2$ volte nella somma, ovvero
			incrementa la somma di $2$
	\end{itemize}
	\hfill$\square$
\end{flushleft}

% $$$$$$$$$$$$$$$$$$$$$$$$$$$$$$$$$$$$$$$$$$$$$$$$$$$$$$$$$$$$$$$$$$$$$$$$$$$$$$$$

\subsection*{teorema: l'algoritmo di ricerca locale \'e $\frac{1}{2}$-approssimante}
\addcontentsline{toc}{subsection}{teorema: l'algoritmo di ricerca locale \'e $\frac{1}{2}$-approssimante}
\begin{flushleft}
	l'algoritmo di ricerca locale \'e $\frac{1}{2}$-approssimante \newline \\
	\vspace{0.5cm}
	\textbf{dimostrazione:}
	\begin{itemize}
		\item mostriamo che ogni ottimo locale ha $(V_1,V_2)$ ha misura:
			$$m\geq\frac{\vert E\vert}{2}$$
		\item ci\'o implica:
			$$\frac{m}{m^*}\geq\frac{\frac{\vert E\vert}{2}}{\vert E\vert}=\frac{1}{2}$$
		\vspace{0.5cm}
		\item poich\'e $m^*\leq\vert E\vert$
		\item dato un ottimo locale $(V_1,V_2)$ denotiamo con $h$ il numero di archi interni, ovvero con entrambi gli estremi in $V_1$ o in $V_2$
		\item chiaramente, $m+h=\vert E\vert$
		\item per ogni nodo $v_i\in V$ definiamo i gradi interni ed esterni del nodo come segue:
		\begin{itemize}
			\item $\delta_i^{int}=$ numero di archi che $v_i$ ha verso i nodi nella sua stessa partizione, ovvero:
				$$\delta_i^{int}=\vert\{v_k\vert\{v_i,v_k\}\in E\text{ e }(v_i,v_k\in V_1)\text{ o }(v_i,v_k\in V_2)\}\vert$$
			\item $\delta_i^{ext}=$ numero di archi che $v_i$ ha verso i nodi nell'altra partizione, ovvero:
				$$\delta_i^{ext}=\vert\{v_k\vert\{v_i,v_k\}\in E\text{ e }(v_i\in V_1,v_k\in V_2)\text{ o }(v_i\in V_2,v_k\in V_1)\}\vert$$
		\end{itemize}
		\item poich\'e la soluzione nell'intorno $(V_{1i},V_{2i})$ ha misura non maggiore ($\leq$) di quella di $(V_1,V_2)$ (ottimo locale), abbiamo:
			$$m-\delta_i^{ext}+\delta_i^{int}\leq m$$
		\item e quindi:
			$$\delta_i^{int}-\delta_i^{ext}\leq 0$$
		\item riassumento, su tutti i nodi, abbiamo:
			$$\sum_{v_i\in V}\delta_i^{int}-\sum_{v_i\in V}\delta_i^{ext}=\sum_{v_i\in V}(\delta_i^{int}-\delta_i^{ext})\leq 0$$
		\item dal fatto precedente:
			$$\sum_{v_i\in V}\delta_i^{int}=2h$$
			(perch\'e \'e come sommare i gradi dei nodi del grafo contenente solo gli archi interni)
		\item e (sempre dal fatto precedente):
			$$\sum_{v_i\in V}\delta_i^{ext}=2m$$
			(perch\'e \'e come sommare i gradi dei nodi del grafo contenente solo gli archi esterni)
		\item quindi:
			$$0\geq\sum_{v_i\in V}\delta_i^{int}-\sum_{v_i\in V}\delta_i^{ext}=2h-2m$$
		\item ovvero $m\geq h$
		\item quindi (aggiungendo $m$ su entrambi i lati e dividendo per $2$), otteniamo:
			$$\color{gray}\frac{2m}{2}\geq\frac{(m+h)}{2}=\color{black}m\geq\frac{(m+h)}{2}=\frac{\vert E\vert}{2}$$
	\end{itemize}
	\hfill$\square$
\end{flushleft}

% $$$$$$$$$$$$$$$$$$$$$$$$$$$$$$$$$$$$$$$$$$$$$$$$$$$$$$$$$$$$$$$$$$$$$$$$$$$$$$$$

\subsection*{TODO: esempio esecuzione algoritmo di ricerca locale su grafo}
\addcontentsline{toc}{subsection}{TODO: esempio esecuzione algoritmo di ricerca locale su grafo}
\begin{flushleft}
\end{flushleft}

% $$$$$$$$$$$$$$$$$$$$$$$$$$$$$$$$$$$$$$$$$$$$$$$$$$$$$$$$$$$$$$$$$$$$$$$$$$$$$$$$

\subsection*{conclusioni sulla tecnica della ricerca locale}
\addcontentsline{toc}{subsection}{conclusioni sulla tecnica della ricerca locale}
\begin{flushleft}
	\begin{itemize}
		\item come gli algoritmi greedy, gli algoritmi di ricerca locale hanno buone performance
			nella pratica e portano alla determinazione di buone euristiche (algoritmi che eseguono
			bene nella pratica ma che di solito non hanno prestazioni garantite in termini di tempo
			o approssimazione)
	\end{itemize}
\end{flushleft}

% $$$$$$$$$$$$$$$$$$$$$$$$$$$$$$$$$$$$$$$$$$$$$$$$$$$$$$$$$$$$$$$$$$$$$$$$$$$$$$$$

\newpage
