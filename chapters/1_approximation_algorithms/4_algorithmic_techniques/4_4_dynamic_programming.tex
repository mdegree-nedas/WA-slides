\section*{algorithmic techniques: dynamic programming (part 1)}
\addcontentsline{toc}{section}{algorithmic techniques: dynamic programming (part 1)}

% $$$$$$$$$$$$$$$$$$$$$$$$$$$$$$$$$$$$$$$$$$$$$$$$$$$$$$$$$$$$$$$$$$$$$$$$$$$$$$$$

\subsection*{caratteristiche}
\addcontentsline{toc}{subsection}{caratteristiche}
\begin{flushleft}
	\begin{itemize}
		\item come nel paradigma divide-and-conquer, suddividi il problema in sottoproblemi pi\'u piccoli, risolvi ricorsivamente ciasun sottoproblema e combina le soluzioni dei sottoproblemi per formare la soluzione al problema originale
		\item ricorrenza facile da calcolare che consente di determinare la soluzione ad un sottoproblema dalla soluzione di sottoproblemi pi\'u piccoli
		\item differentemente da divide-and-conquer, i sottoproblemi non sono indipendenti, ma si sovrappongono, ovvero durante le decomposizioni occorrono frequentemente gli stessi sottoproblemi
		\item idea: ciascun sottoproblema viene risolto solo una volta, ci\'o riduce la complessit\'a temporale
		\item differentemente da divide-and-conquer, di solito \'e con approccio bottom-up invece che top-down, ovvero partendo da sottoproblemi pi\'u piccoli e risolvendo progressivamente quelli pi\'u grandi, fino al problema iniziale
	\end{itemize}
\end{flushleft}

% $$$$$$$$$$$$$$$$$$$$$$$$$$$$$$$$$$$$$$$$$$$$$$$$$$$$$$$$$$$$$$$$$$$$$$$$$$$$$$$$

\subsection*{uno sguardo pi\'u ravvicinato...}
\addcontentsline{toc}{subsection}{uno sguardo pi\'u ravvicinato...}
\begin{flushleft}
	\begin{itemize}
		\item il paradigma divide-and-conquer \'e basato sulla decomposzione dei problemi in sottoproblemi pi\'u piccoli:
		\begin{itemize}
			\item risolvi ricorsivamente i sottoproblemi
			\item combina le soluzioni dei sottoproblemi per determinare la soluzione del problema iniziale
		\end{itemize}
		\item se un problema di taglia $n$ \'e decomposto in $k$ sottoproblemi di taglie $n_1,n_2,\ldots,n_k<n$, rispettivamente, allora la complessit\'a temporale pu\'o essere espressa dall ricorrenza
			$$T(n)=T(n_1)+T(n_2)+\ldots+T(n_k)+C(n)$$
		con $C(n)=$ tempo per combinare le $k$ sottosoluzioni
		\item la ricorrenza pu\'o essere risolta con metodi differenti, come ad esempio il ricorso al celebre Master Theorem
		\item un classico esempio di applicazione di divide-and-conquer \'e il calcolo dei numeri di Fibonacci
		\item l'algoritmo deriva direttamente dalla definizione ricorsiva di tali numeri
	\end{itemize}
\end{flushleft}

% $$$$$$$$$$$$$$$$$$$$$$$$$$$$$$$$$$$$$$$$$$$$$$$$$$$$$$$$$$$$$$$$$$$$$$$$$$$$$$$$

\subsection*{algoritmo: Fibonacci}
\addcontentsline{toc}{subsection}{algoritmo: Fibonacci}
\begin{flushleft}
	\begin{itemize}
		\item caso base: ($n\leq 2$) $F(1)=F(2)=1$
		\item caso induttivo: ($n>2$) $F(n)=F(n-1)+F(n-2),n$
	\end{itemize}
	\begin{algorithm}
		\caption{Fibonacci}
		\begin{algorithmic}
			\IF{$n=1\text{ or }n=2$}
				\RETURN 1
			\ELSE
				\RETURN $Fibonacci(n-1)+Fibonacci(n-2)$
			\ENDIF
		\end{algorithmic}
	\end{algorithm}
	\begin{itemize}
		\item complessit\'a temporale:
			$$T(n)=T(n-1)+T(n-2)+\Theta(1)$$
		\item che restituisce:
			$$T(n)=O(2^n)$$
		\item albero delle chiamate ricorsive:
		\begin{itemize}
			\item nota:
			\begin{itemize}
				\item inefficiente: gli stessi sottoproblemi vengono risolti ripetutamente per molte volte
			\end{itemize}
		\end{itemize}
	\end{itemize}
\end{flushleft}

% $$$$$$$$$$$$$$$$$$$$$$$$$$$$$$$$$$$$$$$$$$$$$$$$$$$$$$$$$$$$$$$$$$$$$$$$$$$$$$$$

\subsection*{algoritmo: Fibonacci 2}
\addcontentsline{toc}{subsection}{algoritmo: Fibonacci 2}
\begin{flushleft}
	\begin{itemize}
		\item programmazione dinamica:
		\begin{itemize}
			\item memorizza la soluzione di ciascun sottoproblema in una tabella o in un array, cos\'i da evitare di risolverli ripetutamente
			\item nell'algoritmo risultante, $F$ \'e un array esterno globale visibile a tutte le chiamate ricorsive
		\end{itemize}
		\item nuovo albero delle chiamate ricorsive
	\end{itemize}
	\begin{algorithm}
		\caption{Fibonacci 2}
		\begin{algorithmic}
			\IF{$n=1\text{ or }n=2$}
				\STATE $F[n]=1$
				\RETURN $F[n]$
			\ELSE
				\IF{$F[n]$ \'e stato gi\'a assegnato}
					\RETURN $F[n]$
				\ELSE
					\STATE $F[n]=Fibonacci(n-1)+Fibonacci(n-2)$
					\RETURN $F[n]$
				\ENDIF
			\ENDIF
		\end{algorithmic}
	\end{algorithm}
\end{flushleft}

% $$$$$$$$$$$$$$$$$$$$$$$$$$$$$$$$$$$$$$$$$$$$$$$$$$$$$$$$$$$$$$$$$$$$$$$$$$$$$$$$

\subsection*{algoritmo: Fibonacci 3}
\addcontentsline{toc}{subsection}{algoritmo: Fibonacci 3}
\begin{flushleft}
	\begin{algorithm}
		\caption{Fibonacci 3}
		\begin{algorithmic}
			\STATE $F[1]=1$
			\STATE $F[2]=1$
			\FOR{$i=3$ to $n$}
				\STATE $F[i]=F[i-1]+F[i-2]$
			\ENDFOR
			\RETURN $F[n]$
		\end{algorithmic}
	\end{algorithm}
\end{flushleft}

% $$$$$$$$$$$$$$$$$$$$$$$$$$$$$$$$$$$$$$$$$$$$$$$$$$$$$$$$$$$$$$$$$$$$$$$$$$$$$$$$

\subsection*{riassumendo}
\addcontentsline{toc}{subsection}{riassumendo}
\begin{flushleft}
	\begin{itemize}
		\item in programmazione dinamica:
		\begin{itemize}
			\item il problema iniziale pu\'o essere ricorsivamente decomposto in sottoproblemi
			\item gli stessi sottoproblemi occorrono molte volte e sono risolti una volta soltanto
			\item la soluzione di un sottoproblema pu\'o essere ottenuta combinando quelle dei sottoproblemi pi\'u piccoli
		\end{itemize}
		\item 2 possibili implementazioni:
		\begin{itemize}
			\item top-down (con annotazione in tabella)
			\item bottom-up
		\end{itemize}
	\end{itemize}
\end{flushleft}

% $$$$$$$$$$$$$$$$$$$$$$$$$$$$$$$$$$$$$$$$$$$$$$$$$$$$$$$$$$$$$$$$$$$$$$$$$$$$$$$$

\subsection*{top-down vs. bottom-up}
\addcontentsline{toc}{subsection}{top-down vs. bottom-up}
\begin{flushleft}
	\begin{itemize}
		\item top-down
		\begin{itemize}
			\item sfrutta l'annotazione in tabella
			\item PRO: risolve solo i sottoproblemi strettamente necessari
			\item CON: overhead derivante dalla catena di chiamate ricorsive
		\end{itemize}
		\item bottom-up
		\begin{itemize}
			\item \'e la scelta tipica nella programmazione dinamica
			\item PRO: \'e in ogni caso generalmente pi\'u efficiente perch\'e elimina il peso della ricorsione, il quale incide maggiormente sulle prestazioni
			\item CON: risolve anche i problemi non necessari
		\end{itemize}
	\end{itemize}
\end{flushleft}

% $$$$$$$$$$$$$$$$$$$$$$$$$$$$$$$$$$$$$$$$$$$$$$$$$$$$$$$$$$$$$$$$$$$$$$$$$$$$$$$$

\subsection*{divide-and-conquer vs. dynamic programming}
\addcontentsline{toc}{subsection}{divide-and-conquer vs. dynamic programming}
\begin{flushleft}
	\begin{itemize}
		\item divide-and-conquer
		\begin{itemize}
			\item tecnica ricorsiva
			\item approccio top-down (problemi divisi in sottoproblemi)
			\item utile quando i sottoproblemi sono indipendenti (ovvero differenti)
			\item altrimenti, gli stessi sottoproblemi vengono risolti pi\'u volte
		\end{itemize}
		\item dynamic programming
		\begin{itemize}
			\item tecnica iterativa
			\item tipicamente approccio bottom-up
			\item utile quando i sottoproblemi si sovrapppongono (ovvero coincidono)
			\item ciasun sottoproblema viene risolto una volta soltanto
		\end{itemize}
	\end{itemize}
\end{flushleft}

% $$$$$$$$$$$$$$$$$$$$$$$$$$$$$$$$$$$$$$$$$$$$$$$$$$$$$$$$$$$$$$$$$$$$$$$$$$$$$$$$

\section*{algorithmic techniques: dynamic programming (part 2)}
\addcontentsline{toc}{section}{algorithmic techniques: dynamic programming (part 2)}

% $$$$$$$$$$$$$$$$$$$$$$$$$$$$$$$$$$$$$$$$$$$$$$$$$$$$$$$$$$$$$$$$$$$$$$$$$$$$$$$$

\subsection*{progettazione di algoritmi di programmazione dinamica}
\addcontentsline{toc}{subsection}{progettazione di algoritmi di programmazione dinamica}
\begin{flushleft}
	\begin{itemize}
		\item fornire una decomposizione ricorsiva dei sottoproblemi
		\item calcolare le sottosoluzioni in maniera bottom-up, ovvero partendo dai sottoproblemi di taglia pi\'u piccola
		\begin{itemize}
			\item utilizzare una tabella per memorizzare i risultati dei sottoproblemi
			\item evitare il calcolo delle stesse soluzioni sfruttando la tabella
		\end{itemize}
		\item combinare le soluzioni dei sottoproblemi gi\'a risolti per costruire quelle dei sottoproblemi di taglia maggiore, fino alla risoluzione del problema originale
	\end{itemize}
\end{flushleft}

% $$$$$$$$$$$$$$$$$$$$$$$$$$$$$$$$$$$$$$$$$$$$$$$$$$$$$$$$$$$$$$$$$$$$$$$$$$$$$$$$

\subsection*{complessit\'a degli algoritmi di programmazione dinamica}
\addcontentsline{toc}{subsection}{complessit\'a degli algoritmi di programmazione dinamica}
\begin{flushleft}
	\begin{itemize}
		\item consideriamo la seguente tabella:
		\begin{itemize}
			\item $n=$ taglia dei sottoproblemi ($1,2,\ldots,n$)
			\item $k=$ parametri dei sottoproblemi ($p_1,p_2,\ldots,p_k$)
		\end{itemize}
		\item taglia della tabella $=$ numero di sottoproblemi $=nk$
		\item complessit\'a:
		\begin{itemize}
			\item $[\text{ taglia della tabella }]\times[\text{ tempo per combinare le soluzioni }]$
			\item il tempo per combinare le soluzioni \'e sempre banalmente polinomiale
			\item la complessit\'a \'e polinomale se la tabella ha taglia polinomale, ovvero se \'e presente un numero polinomale di differenti sottoproblemi
		\end{itemize}
	\end{itemize}
\end{flushleft}

% $$$$$$$$$$$$$$$$$$$$$$$$$$$$$$$$$$$$$$$$$$$$$$$$$$$$$$$$$$$$$$$$$$$$$$$$$$$$$$$$

\subsection*{problema: max 0-1 knapsack (gi\'a definito precedentemente)}
\addcontentsline{toc}{subsection}{problema: max 0-1 knapsack (gi\'a definito precedentemente)}
\begin{flushleft}
	\begin{itemize}
		\item INPUT:
		\begin{itemize}
			\item un insieme finito di oggetti $O$
			\item un profitto intero $p_i$, $\forall o_i\in O$
			\item un peso intero $w_i$, $\forall o_i\in O$
			\item un intero positivo $b$ ($b>0$)
		\end{itemize}
		\item SOLUZIONE:
		\begin{itemize}
			\item un sottoinsieme di oggetti $Q\subseteq O$ tale che $\sum_{o_i\in Q}w_i\leq b$
		\end{itemize}
		\item MISURA:
		\begin{itemize}
			\item profitto totale degli oggetti scelti, ovvero $\sum_{o_i\in Q}p_i$
		\end{itemize}
		\vspace{0.5cm}
		\item senza perdere di generalit\'a, in seguito, assumeremo sempre che:
		\begin{itemize}
			\item $w_i\leq b$, $\forall o_i\in O$
			\item $p_i>0$, $\forall o_i\in O$
		\end{itemize}
	\end{itemize}
\end{flushleft}

% $$$$$$$$$$$$$$$$$$$$$$$$$$$$$$$$$$$$$$$$$$$$$$$$$$$$$$$$$$$$$$$$$$$$$$$$$$$$$$$$

\subsection*{algoritmo brute force}
\addcontentsline{toc}{subsection}{algoritmo brute force}
\begin{flushleft}
	\begin{itemize}
		\item semplice algoritmo che enumera tutti i possibili $2^n$ sottoinsiemi degli $n$ elementi
		\item sceglie la migliore combinazione (miglior profitto)
		\item l'algoritmo di programmazione solitamente ha performance migliori
	\end{itemize}
\end{flushleft}

% $$$$$$$$$$$$$$$$$$$$$$$$$$$$$$$$$$$$$$$$$$$$$$$$$$$$$$$$$$$$$$$$$$$$$$$$$$$$$$$$

\subsection*{progettazione dell'algoritmo di programmazione dinamica}
\addcontentsline{toc}{subsection}{progettazione dell'algoritmo di programmazionedinamica}
\begin{flushleft}
	\begin{itemize}
		\item definizione:
		\begin{itemize}
			\item $OPT(i,w)=$ sottoinsieme con profitto massimo di oggetti $1,2,\ldots,i$ con limite di peso $w$
		\end{itemize}
		\item fatto:
		\begin{itemize}
			\item $OPT(n,b)=$ soluzione ottima del problema iniziale
		\end{itemize}
		\item le seguente alternative possono occorrere per $OPT(i,w)$:
		\begin{enumerate}
			\item $OPT$ non seleziona l'oggetto $i$
			\begin{itemize}
				\item $OPT$ seleziona il migliore tra $\{1,2,\ldots,i-1\}$ utilizzando il limite di peso $w$
			\end{itemize}
			\item $OPT$ seleziona l'oggetto $i$
			\begin{itemize}
				\item $OPT$ seleziona il migliore tra $\{1,2,\ldots,i-1\}$ utilizzando il limite di peso $w-w_i$
			\end{itemize}
		\end{enumerate}
		\item assumiamo che $OPT(k,w)$ sia la soluzione ottima per gli elementi $\{o_1,o_2,\ldots,o_k\}$
		\item nota: la soluzione ottima $OPT(k+1,w)$ potrebbe non corrispondere a $OPT(k,w)$
		\item anche perch\'e $OPT(k+1,w)$ potrebbe non essere un superset di $OPT(k,w)$
	\end{itemize}
\end{flushleft}

% $$$$$$$$$$$$$$$$$$$$$$$$$$$$$$$$$$$$$$$$$$$$$$$$$$$$$$$$$$$$$$$$$$$$$$$$$$$$$$$$

\subsection*{definizione ricorsiva per $OPT$}
\addcontentsline{toc}{subsection}{definizione ricorsiva per $OPT$}
\begin{flushleft}
	\begin{itemize}
		\item possiamo dunque fornire la seguente definizione ricorsiva per $OPT$:
		\begin{itemize}
			\item $OPT(i,w)=\emptyset$ se $i=0$
			\item $OPT(i,w)=OPT(i-1,w)$ se $w_i>w$
			\item $OPT(i,w)=$ scelta migliore tra $OPT(i-1,w)$ e $OPT(i-1,w-w_i)\cup\{o_i\}$ (altrimenti)
		\end{itemize}
	\end{itemize}
\end{flushleft}

% $$$$$$$$$$$$$$$$$$$$$$$$$$$$$$$$$$$$$$$$$$$$$$$$$$$$$$$$$$$$$$$$$$$$$$$$$$$$$$$$

\subsection*{definizione ricorsiva per la misura $m$ della soluzione ottima $OPT(i,w)$}
\addcontentsline{toc}{subsection}{definizione ricorsiva per la misura $m$ della soluzione ottima $OPT(i,w)$}
\begin{flushleft}
	\begin{itemize}
		\item in termini di misura $m(i,w)$ della soluzione ottima $OPT(i,w)$
		\begin{itemize}
			\item $m(i,w)=0$ se $i=0$
			\item $m(i,w)=m(i-1,w)$ se $w_i>w$
			\item $m(i,w)=\max\{m(i-1,w),m(i-1,w-w_i)+p_i\}$ (altrimenti)
		\end{itemize}
		\item chiaramente, $m^*=m(n,b)$
	\end{itemize}
\end{flushleft}

% $$$$$$$$$$$$$$$$$$$$$$$$$$$$$$$$$$$$$$$$$$$$$$$$$$$$$$$$$$$$$$$$$$$$$$$$$$$$$$$$

\subsection*{riepilogo definizioni ricorsive per $m$ e $OPT$}
\addcontentsline{toc}{subsection}{riepilogo definizioni ricorsive per $m$ e $OPT$}
\begin{flushleft}
	\begin{itemize}
		\item come risultato, questo significa che il miglior sottoinsieme di $k$ oggetti con vincolo di peso $w$ \'e (mutua esclusione):
		\begin{itemize}
			\item il miglior sottoinsieme di $(k-1)$ oggetti con peso totale $w$
			\item il miglior sottoinsieme di $(k-1)$ oggetti con peso totale $w-w_k$, pi\'u il contributo (il suo profitto) del $k$-esimo oggetto
			\item quindi per quando riguarda la seguente formula ricorsiva:
			\begin{itemize}
				\item $OPT(i,w)=\emptyset$ se $i=0$
				\item $OPT(i,w)=OPT(i-1,w)$ se $w_i>w$
				\item $OPT(i,w)=$ scelta migliore tra $OPT(i-1,w)$ e $OPT(i-1,w-w_i)\cup\{o_i\}$ (altrimenti)
			\end{itemize}
		\item o il $k$-esimo oggetto non pu\'o essere parte della soluzione (poich\'e il suo solo peso \'e cos\'i grande che l'oggetto stesso non entra nel knapsack)
		\item altrimenti, scegliamo la soluzione migliore tra:
		\begin{itemize}
			\item la soluzione che include il nuovo oggetto
			\item la soluzione migliore che non include il nuovo oggetto
		\end{itemize}
		\end{itemize}
	\end{itemize}
\end{flushleft}

% $$$$$$$$$$$$$$$$$$$$$$$$$$$$$$$$$$$$$$$$$$$$$$$$$$$$$$$$$$$$$$$$$$$$$$$$$$$$$$$$

\subsection*{algoritmo: Progr-Dyn-Knapsack}
\addcontentsline{toc}{subsection}{algoritmo: Progr-Dyn-Knapsack}
\begin{flushleft}
	\begin{algorithm}
		\caption{Progr-Dyn-Knapsack}
		\begin{algorithmic}
			\STATE \color{gray} // inizializzazione a $0$ della prima riga dell'array bidimensionale $M$ (da $1$ a $b$) \color{black}
			\FOR{$w=1$ to $b$}
				\STATE $M[0,w]=0$
			\ENDFOR
			\STATE \color{gray} // inizializzazione a $0$ della prima colonna dell'array bidimensionale $M$ (da $0$ a $n$) \color{black}
			\FOR{$i=0$ to $n$}
				\STATE $M[i,0]=0$
			\ENDFOR
			\FOR{$i=1$ to $n$}
				\FOR{$w=1$ to $b$}
					\IF{$w_i>w$}
						\STATE $M[i,w]=M[i-1,w]$
					\ELSE
						\STATE $M[i,w]=\max\{M[i-1,w],M[i-1,w-w_i]+p_i\}$
					\ENDIF
				\ENDFOR
			\ENDFOR
			\RETURN $M[n,b]$
		\end{algorithmic}
	\end{algorithm}
	\begin{itemize}
		\item l'algoritmo Progr-Dyn-Knapsack per il problema max 0-1 knapsack, trova il massimo valore che pu\'o essere inserito dentro il knapsack
		\item il valore viene memorizzato in $M[n,b]$ al termine della procedura
		\item per scoprire quali sono gli oggetti che sono stati inseriti nella soluzione ottima,bisogna tornare indietro nella tabella:
		\begin{itemize}
			\item dobbiamo memorizzare in qualche modo ciascun oggetto aggiunto
		\end{itemize}
	\end{itemize}
\end{flushleft}

% $$$$$$$$$$$$$$$$$$$$$$$$$$$$$$$$$$$$$$$$$$$$$$$$$$$$$$$$$$$$$$$$$$$$$$$$$$$$$$$$

\subsection*{algoritmo: Progr-Dyn-Knapsack (trovare gli oggetti inseriti)}
\addcontentsline{toc}{subsection}{algoritmo: Progr-Dyn-Knapsack (trovare gli oggetti inseriti)}
\begin{flushleft}
	\begin{algorithm}
		\caption{Progr-Dyn-Knapsack (trovare gli oggetti inseriti)}
		\begin{algorithmic}
			\STATE $i=n$
			\STATE $k=w$
			\IF{$M[i,k]\neq M[i-1,k]$}
				\STATE marca l'oggetto $i$ come 'inserito nel knapsack'
				\STATE $i=i-1$
				\STATE $k=k-w_i$
			\ELSE
				\STATE \color{gray} // assumi che l'oggetto $i$-esimo non sia stato inserito nel knapsack \color{black}
				\STATE $i=i-1$
			\ENDIF
		\end{algorithmic}
	\end{algorithm}
\end{flushleft}

% $$$$$$$$$$$$$$$$$$$$$$$$$$$$$$$$$$$$$$$$$$$$$$$$$$$$$$$$$$$$$$$$$$$$$$$$$$$$$$$$

\subsection*{teorema: l'algoritmo Progr-Dyn-Knapsack ha complessit\'a temporale $O(nb)$}
\addcontentsline{toc}{subsection}{teorema: l'algoritmo Progr-Dyn-Knapsack ha complessit\'a temporale $O(nb)$}
\begin{flushleft}
	la complessit\'a temporale dell'algoritmo Progr-Dyn-Knapsack \'e $O(nb)$ \newline \\
	\textbf{dimostrazione:}
	\begin{itemize}
		\item l'algoritmo impiega $O(1)$ per ciascuna entry della tabella
		\item vi sono $O(nb)$ entry nella tabella
		\item dopo aver calcolato i valori possiamo risalire per trovare la soluzione ottima:
		\begin{itemize}
			\item prendi l'elemento $o_i$ in $OPT(i,w)\iff M[i-1,w]<M[i,w]$
		\end{itemize}
	\end{itemize}
	\hfill$\square$
\end{flushleft}

% $$$$$$$$$$$$$$$$$$$$$$$$$$$$$$$$$$$$$$$$$$$$$$$$$$$$$$$$$$$$$$$$$$$$$$$$$$$$$$$$

\subsection*{domanda: l'algoritmo Progr-Dyn-Knapsack \'e polinomiale?}
\addcontentsline{toc}{subsection}{domanda: l'algoritmo Progr-Dyn-Knapsack \'e polinomiale?}
\begin{flushleft}
	l'algoritmo Progr-Dyn-Knapsack \'e polinomiale? \newline
	(suggerimento: considera il caso in cui $b=2^n$) \newline \\
	\textbf{risposta:}
	\begin{itemize}
		\item per essere polinomiale, la complessit\'a dovrebbe essere polinomiale nel logaritmo dei valori codificati nell'istanza in input, ovvero rispetto a $\log b$
		\item questa complessit\'a \'e chiamata \textbf{pseudo-polinomale}, ovvero polinomale nella dimensione e nei valori in input, non solo nella dimensione dell'input
	\end{itemize}
\end{flushleft}

% $$$$$$$$$$$$$$$$$$$$$$$$$$$$$$$$$$$$$$$$$$$$$$$$$$$$$$$$$$$$$$$$$$$$$$$$$$$$$$$$

\subsection*{algoritmo Progr-Dyn-Knapsack-Dual: approccio duale}
\addcontentsline{toc}{subsection}{algoritmo Progr-Dyn-Knapsack-Dual: approccio duale}
\begin{flushleft}
	\begin{itemize}
		\item definizione:
		\begin{itemize}
			\item $OPT(i,p)=$ sottoinsieme di peso minimo di oggetti $1,2,\ldots,i$ con profitto almeno pari a $p$ ($\geq$)
		\end{itemize}
		\item domanda: quale sottoproblema corrisponde alla soluzione ottima?
		\item le seguente alternative possono occorrere per $OPT(i,p)$:
		\begin{enumerate}
			\item $OPT$ non seleziona l'oggetto $i$
			\begin{itemize}
				\item $OPT$ seleziona il migliore tra $\{1,2,\ldots,i-1\}$ utilizzando il limite di profitto $p$
			\end{itemize}
			\item $OPT$ seleziona l'oggetto $i$
			\begin{itemize}
				\item $OPT$ seleziona il migliore tra $\{1,2,\ldots,i-1\}$ utilizzando il limite di profitto $p-p_i$
			\end{itemize}
		\end{enumerate}
	\end{itemize}
\end{flushleft}

% $$$$$$$$$$$$$$$$$$$$$$$$$$$$$$$$$$$$$$$$$$$$$$$$$$$$$$$$$$$$$$$$$$$$$$$$$$$$$$$$

\subsection*{definizione ricorsiva per $OPT$ (duale)}
\addcontentsline{toc}{subsection}{definizione ricorsiva per $OPT$ (duale)}
\begin{flushleft}
	\begin{itemize}
		\item possiamo dunque fornire la seguente definizione ricorsiva per $OPT$:
		\begin{itemize}
			\item $OPT(i,p)=$ non definito se $i=0$
			\item $OPT(i,p)=$ scelta migliore tra $OPT(i-1,p)$ e $\{o_i\}$ se $p_i\geq p$
			\item $OPT(i,p)=$ scelta migliore tra $OPT(i-1,p)$ e $OPT(i-1,p-p_i)\cup\{o_i\}$ (altrimenti, non definito se entrambi non sono definiti)
		\end{itemize}
	\end{itemize}
\end{flushleft}

% $$$$$$$$$$$$$$$$$$$$$$$$$$$$$$$$$$$$$$$$$$$$$$$$$$$$$$$$$$$$$$$$$$$$$$$$$$$$$$$$

\subsection*{definizione ricorsiva per la misura $v$ della soluzione ottima $OPT(i,p)$}
\addcontentsline{toc}{subsection}{definizione ricorsiva per la misura $m$ della soluzione ottima $OPT(i,w)$}
\begin{flushleft}
	\begin{itemize}
		\item in termini di peso $v(i,p)$ della soluzione ottima $OPT(i,p)$
		\begin{itemize}
			\item $v(i,p)=\infty$ se $i=0$
			\item $v(i,p)=\min\{v(i-1,p),w_i\}$ se $p_i\geq p$
			\item $v(i,p)=\min\{v(i-1,p),v(i-1,p-p_i)+w_i\}$ (altrimenti)
		\end{itemize}
	\end{itemize}
\end{flushleft}

% $$$$$$$$$$$$$$$$$$$$$$$$$$$$$$$$$$$$$$$$$$$$$$$$$$$$$$$$$$$$$$$$$$$$$$$$$$$$$$$$

\newpage
\subsection*{algoritmo: Progr-Dyn-Knapsack-Dual}
\addcontentsline{toc}{subsection}{algoritmo: Progr-Dyn-Knapsack-Dual}
\begin{flushleft}
	\begin{algorithm}
		\caption{Progr-Dyn-Knapsack-Dual}
		\begin{algorithmic}
			\STATE \color{gray} // inizializzazione a $\infty$ della prima riga dell'array bidimensionale $V$ (da $1$ a $P$) \color{black}
			\FOR{$p=1$ to $P$}
				\STATE $V[0,p]=\infty$
			\ENDFOR
			\STATE \color{gray} // inizializzazione a $\infty$ della prima colonna dell'array bidimensionale $V$ (da $0$ a $n$) \color{black}
			\FOR{$i=0$ to $n$}
				\STATE $V[i,0]=\infty$
			\ENDFOR
			\FOR{$i=1$ to $n$}
				\FOR{$p=1$ to $P$}
					\IF{$p_i\geq p$}
						\STATE $V[i,p]=\min\{V[i-1,p],w_i\}$
					\ELSE
						\STATE $V[i,p]=\min\{V[i-1,p],V[i-1,p-p_i]+w_i\}$
					\ENDIF
				\ENDFOR
			\ENDFOR
			\RETURN $\max p\text{ }\vert\text{ }V[n,p]\leq b$
		\end{algorithmic}
	\end{algorithm}
	\begin{itemize}
		\item problema: come dovremmo scegliere $P$?
		\item risposta: grande abbastanza da includere l'ottimo, ovvero qualunque $upper\text{ }bound$ di $m^*$, cio\'e $P\geq m^*$
		\item scelta:
			$$P=np_{\max}\geq\sum_{i=1}^np_i\geq m^*\text{, }(p_{\max}=\max p_j)$$
	\end{itemize}
\end{flushleft}

% $$$$$$$$$$$$$$$$$$$$$$$$$$$$$$$$$$$$$$$$$$$$$$$$$$$$$$$$$$$$$$$$$$$$$$$$$$$$$$$$

\subsection*{teorema: l'algoritmo Progr-Dyn-Knapsack-Dual ha complessit\'a temporale $O(n^2p_{\max})$}
\addcontentsline{toc}{subsection}{teorema: l'algoritmo Progr-Dyn-Knapsack-Dual ha complessit\'a temporale $O(n^2p_{\max})$}
\begin{flushleft}
	l'algoritmo Progr-Dyn-Knapsack-Dual ha complessit\'a temporale $O(n^2p_{\max})$ \newline \\
	\textbf{dimostrazione:}
	\begin{itemize}
		\item l'algoritmo impiega $O(1)$ per ogni entry della tabella
		\item vi sono $O(nP)=O(n^2p_{\max})$ entry nella tabella
		\item dopo il calcolo del valori, possiamo risalire per trovare la soluzione ottima:
		\begin{itemize}
			\item prendi l'elemento $o_i$ in $OPT(i,p)\iff V[i-1,p]>V[i,p]$
		\end{itemize}
	\end{itemize}
	\hfill$\square$
\end{flushleft}

% $$$$$$$$$$$$$$$$$$$$$$$$$$$$$$$$$$$$$$$$$$$$$$$$$$$$$$$$$$$$$$$$$$$$$$$$$$$$$$$$

\newpage
