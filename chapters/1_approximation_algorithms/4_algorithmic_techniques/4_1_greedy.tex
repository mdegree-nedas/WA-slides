\section*{approximation}
\addcontentsline{toc}{section}{algorithmic techniques: greedy}

% $$$$$$$$$$$$$$$$$$$$$$$$$$$$$$$$$$$$$$$$$$$$$$$$$$$$$$$$$$$$$$$$$$$$$$$$$$$$$$$$

\subsection*{caratteristiche}
\addcontentsline{toc}{subsection}{caratteristiche}
\begin{flushleft}
	\begin{itemize}
		\item la soluzione viene determinata in step
		\item ad ogni step l'algoritmo esegue la scelta che sembra essere la migliore in quello step, senza considerare le possibili conseguenze nei futuri step
	\end{itemize}
\end{flushleft}

% $$$$$$$$$$$$$$$$$$$$$$$$$$$$$$$$$$$$$$$$$$$$$$$$$$$$$$$$$$$$$$$$$$$$$$$$$$$$$$$$

\subsection*{problema: max 0-1 knapsack}
\addcontentsline{toc}{subsection}{problema: max 0-1 knapsack}
\begin{flushleft}
	\begin{itemize}
		\item INPUT:
		\begin{itemize}
			\item un insieme finito di oggetti $O$
			\item un profitto intero $p_i$ $\forall o_i\in O$
			\item un volume intero $a_i$ $\forall o_i\in O$
			\item un intero positivo $b$
		\end{itemize}
		\item SOLUZIONE:
		\begin{itemize}
			\item un sottoinsieme di oggetti $Q\subseteq O$ tale che $\sum_{o_i\in Q}a_i\leq b$
		\end{itemize}
		\item MISURA:
		\begin{itemize}
			\item profitto totale degli oggetti scelti, ovvero $\sum_{o_i\in Q}p_i$
		\end{itemize}
		\vspace{0.5cm}
		\item senza perdere di generalit\'a, in seguito, assumeremo sempre che:
		\begin{itemize}
			\item $a_i\leq b$ $\forall o_i\in O$
			\item $p_i>0$ $\forall o_i\in O$
		\end{itemize}
	\end{itemize}
\end{flushleft}

% $$$$$$$$$$$$$$$$$$$$$$$$$$$$$$$$$$$$$$$$$$$$$$$$$$$$$$$$$$$$$$$$$$$$$$$$$$$$$$$$

\subsection*{max 0-1 knapsack: descrizione della scelta greedy}
\addcontentsline{toc}{subsection}{max 0-1 knapsack: descrizione della scelta greedy}
\begin{flushleft}
	\begin{itemize}
		\item nella scelta greedy:
		\begin{itemize}
			\item non possiamo considerare solo il profitto degli oggetti, in quanto il loro volume potrebbe essere troppo grande
			\item non possiamo considerare solo il il volume degli oggetti, in quanto il loro profitto potrebbe essere troppo basso
		\end{itemize}
		\item idea: consideriamo gli oggetti in base al profitto per unit\'a di volume, ovvero in base al rapporto $$\frac{p_i}{a_i}$$
		\item l'algoritmo greedy seleziona gli oggetti in ordine decrescente di profitto per volume
	\end{itemize}
\end{flushleft}

% $$$$$$$$$$$$$$$$$$$$$$$$$$$$$$$$$$$$$$$$$$$$$$$$$$$$$$$$$$$$$$$$$$$$$$$$$$$$$$$$

\newpage
\subsection*{algoritmo: Greedy-Knapsack}
\addcontentsline{toc}{subsection}{algoritmo: Greedy-Knapsack}
\begin{flushleft}
	\begin{algorithm}
		\caption{Greedy-Knapsack}
		\begin{algorithmic}
			\STATE $Q=\emptyset$
			\STATE $v=0$
			\color{gray}
			\hfill v = volume del sottoinsieme corrente degli oggetti scelti
			\color{black}
			\STATE ordina gli oggetti in ordine decrescente di profitto per volume $\frac{p_i}{a_i}$
			\STATE siano $o_1,\ldots,o_n$ gli oggetti elencati secondo tale ordine
			\FOR{$i=1$ to $n$}
				\IF{$v+a_i\leq b$}
					\STATE $Q=Q\cup\{o_i\}$
					\STATE $v=v+a_i$
				\ENDIF
			\ENDFOR
			\RETURN $Q$
		\end{algorithmic}
	\end{algorithm}
\end{flushleft}

% $$$$$$$$$$$$$$$$$$$$$$$$$$$$$$$$$$$$$$$$$$$$$$$$$$$$$$$$$$$$$$$$$$$$$$$$$$$$$$$$

\subsection*{teorema: $\forall r<1$ Greedy-Knapsack non \'e r-approssimante}
\addcontentsline{toc}{subsection}{teorema: $\forall r<1$ Greedy-Knapsack non \'e r-approssimante}
\begin{flushleft}
	$\forall r<1$ dato, Greedy-Knapsack non \'e r-approssimante \newline \\
	\textbf{dimostrazione:}
	\begin{itemize}
		\item dato un intero $k=\lceil\frac{1}{r}\rceil$, consideriamo la seguente istanza di max 0-1 knapsack
		\item $\forall n\geq 2$
		\begin{itemize}
			\item $b=kn$ \'e il volume del knapsack
			\item $n-1$ oggetti con profitto $p_i=1$ e volume $a_i=1$
			\item $1$ oggetto con profitto $b-1$ e volume $b$
		\end{itemize}
		\item soluzione restituita:
		\begin{itemize}
			\item l'insieme dei primi $n-1$ oggetti, ovvero $m=n-1$
		\end{itemize}
		\item soluzione ottima
		\begin{itemize}
			\item l'insieme contenente solo l'$n$-esimo oggetto, ovvero $$m^*=b-1=kn-1$$
		\end{itemize}
		\vspace{0.5cm}
		\item quindi: $$\frac{m}{m^*}=\frac{n-1}{kn-1}$$
		\item cos\'i che:
			% TODO: last part of the proof %
			$$\frac{m}{m^*}=\frac{n-1}{kn-1}\leq\frac{n-1}{\frac{n}{r}-1}$$
	\end{itemize}
	\hfill$\square$
\end{flushleft}

% $$$$$$$$$$$$$$$$$$$$$$$$$$$$$$$$$$$$$$$$$$$$$$$$$$$$$$$$$$$$$$$$$$$$$$$$$$$$$$$$

\subsection*{}
\addcontentsline{toc}{subsection}{}
\begin{flushleft}
\end{flushleft}

% $$$$$$$$$$$$$$$$$$$$$$$$$$$$$$$$$$$$$$$$$$$$$$$$$$$$$$$$$$$$$$$$$$$$$$$$$$$$$$$$

\subsection*{}
\addcontentsline{toc}{subsection}{}
\begin{flushleft}
\end{flushleft}

% $$$$$$$$$$$$$$$$$$$$$$$$$$$$$$$$$$$$$$$$$$$$$$$$$$$$$$$$$$$$$$$$$$$$$$$$$$$$$$$$

\subsection*{}
\addcontentsline{toc}{subsection}{}
\begin{flushleft}
\end{flushleft}

% $$$$$$$$$$$$$$$$$$$$$$$$$$$$$$$$$$$$$$$$$$$$$$$$$$$$$$$$$$$$$$$$$$$$$$$$$$$$$$$$

\subsection*{}
\addcontentsline{toc}{subsection}{}
\begin{flushleft}
\end{flushleft}

% $$$$$$$$$$$$$$$$$$$$$$$$$$$$$$$$$$$$$$$$$$$$$$$$$$$$$$$$$$$$$$$$$$$$$$$$$$$$$$$$

\subsection*{}
\addcontentsline{toc}{subsection}{}
\begin{flushleft}
\end{flushleft}

% $$$$$$$$$$$$$$$$$$$$$$$$$$$$$$$$$$$$$$$$$$$$$$$$$$$$$$$$$$$$$$$$$$$$$$$$$$$$$$$$

\subsection*{}
\addcontentsline{toc}{subsection}{}
\begin{flushleft}
\end{flushleft}

% $$$$$$$$$$$$$$$$$$$$$$$$$$$$$$$$$$$$$$$$$$$$$$$$$$$$$$$$$$$$$$$$$$$$$$$$$$$$$$$$

\subsection*{}
\addcontentsline{toc}{subsection}{}
\begin{flushleft}
\end{flushleft}

% $$$$$$$$$$$$$$$$$$$$$$$$$$$$$$$$$$$$$$$$$$$$$$$$$$$$$$$$$$$$$$$$$$$$$$$$$$$$$$$$

\subsection*{}
\addcontentsline{toc}{subsection}{}
\begin{flushleft}
\end{flushleft}

% $$$$$$$$$$$$$$$$$$$$$$$$$$$$$$$$$$$$$$$$$$$$$$$$$$$$$$$$$$$$$$$$$$$$$$$$$$$$$$$$

\newpage
