\section*{approximation}
\addcontentsline{toc}{section}{algorithmic techniques: greedy}

% $$$$$$$$$$$$$$$$$$$$$$$$$$$$$$$$$$$$$$$$$$$$$$$$$$$$$$$$$$$$$$$$$$$$$$$$$$$$$$$$

\subsection*{caratteristiche}
\addcontentsline{toc}{subsection}{caratteristiche}
\begin{flushleft}
	\begin{itemize}
		\item la soluzione viene determinata in step
		\item ad ogni step l'algoritmo esegue la scelta che sembra essere la migliore in quello step, senza considerare le possibili conseguenze nei futuri step
	\end{itemize}
\end{flushleft}

% $$$$$$$$$$$$$$$$$$$$$$$$$$$$$$$$$$$$$$$$$$$$$$$$$$$$$$$$$$$$$$$$$$$$$$$$$$$$$$$$

\subsection*{problema: max 0-1 knapsack}
\addcontentsline{toc}{subsection}{problema: max 0-1 knapsack}
\begin{flushleft}
	\begin{itemize}
		\item INPUT:
		\begin{itemize}
			\item un insieme finito di oggetti $O$
			\item un profitto intero $p_i$ $\forall o_i\in O$
			\item un volume intero $a_i$ $\forall o_i\in O$
			\item un intero positivo $b$
		\end{itemize}
		\item SOLUZIONE:
		\begin{itemize}
			\item un sottoinsieme di oggetti $Q\subseteq O$ tale che $\sum_{o_i\in Q}a_i\leq b$
		\end{itemize}
		\item MISURA:
		\begin{itemize}
			\item profitto totale degli oggetti scelti, ovvero $\sum_{o_i\in Q}p_i$
		\end{itemize}
		\vspace{0.5cm}
		\item senza perdere di generalit\'a, in seguito, assumeremo sempre che:
		\begin{itemize}
			\item $a_i\leq b$ $\forall o_i\in O$
			\item $p_i>0$ $\forall o_i\in O$
		\end{itemize}
	\end{itemize}
\end{flushleft}

% $$$$$$$$$$$$$$$$$$$$$$$$$$$$$$$$$$$$$$$$$$$$$$$$$$$$$$$$$$$$$$$$$$$$$$$$$$$$$$$$

\subsection*{max 0-1 knapsack: descrizione della scelta greedy}
\addcontentsline{toc}{subsection}{max 0-1 knapsack: descrizione della scelta greedy}
\begin{flushleft}
	\begin{itemize}
		\item nella scelta greedy:
		\begin{itemize}
			\item non possiamo considerare solo il profitto degli oggetti, in quanto il loro volume potrebbe essere troppo grande
			\item non possiamo considerare solo il il volume degli oggetti, in quanto il loro profitto potrebbe essere troppo basso
		\end{itemize}
		\item idea: consideriamo gli oggetti in base al profitto per unit\'a di volume, ovvero in base al rapporto $$\frac{p_i}{a_i}$$
		\item l'algoritmo greedy seleziona gli oggetti in ordine decrescente di profitto per volume
	\end{itemize}
\end{flushleft}

% $$$$$$$$$$$$$$$$$$$$$$$$$$$$$$$$$$$$$$$$$$$$$$$$$$$$$$$$$$$$$$$$$$$$$$$$$$$$$$$$

\newpage
\subsection*{algoritmo: Greedy-Knapsack}
\addcontentsline{toc}{subsection}{algoritmo: Greedy-Knapsack}
\begin{flushleft}
	\begin{algorithm}
		\caption{Greedy-Knapsack}
		\begin{algorithmic}
			\STATE $Q=\emptyset$
			\STATE $v=0$
			\color{gray}
			\hfill v = volume del sottoinsieme corrente degli oggetti scelti
			\color{black}
			\STATE ordina gli oggetti in ordine decrescente di profitto per volume $\frac{p_i}{a_i}$
			\STATE siano $o_1,\ldots,o_n$ gli oggetti elencati secondo tale ordine
			\FOR{$i=1$ to $n$}
				\IF{$v+a_i\leq b$}
					\STATE $Q=Q\cup\{o_i\}$
					\STATE $v=v+a_i$
				\ENDIF
			\ENDFOR
			\RETURN $Q$
		\end{algorithmic}
	\end{algorithm}
\end{flushleft}

% $$$$$$$$$$$$$$$$$$$$$$$$$$$$$$$$$$$$$$$$$$$$$$$$$$$$$$$$$$$$$$$$$$$$$$$$$$$$$$$$

\subsection*{teorema: $\forall r<1$ Greedy-Knapsack non \'e r-approssimante}
\addcontentsline{toc}{subsection}{teorema: $\forall r<1$ Greedy-Knapsack non \'e r-approssimante}
\begin{flushleft}
	$\forall r<1$ dato, Greedy-Knapsack non \'e r-approssimante \newline \\
	\textbf{dimostrazione:}
	\begin{itemize}
		\item dato un intero $k=\lceil\frac{1}{r}\rceil$, consideriamo la seguente istanza di max 0-1 knapsack
		\item $\forall n\geq 2$
		\begin{itemize}
			\item $b=kn$ \'e il volume del knapsack
			\item $n-1$ oggetti con profitto $p_i=1$ e volume $a_i=1$
			\item $1$ oggetto con profitto $b-1$ e volume $b$
		\end{itemize}
		\item soluzione restituita:
		\begin{itemize}
			\item l'insieme dei primi $n-1$ oggetti, ovvero $m=n-1$
		\end{itemize}
		\item soluzione ottima
		\begin{itemize}
			\item l'insieme contenente solo l'$n$-esimo oggetto, ovvero $$m^*=b-1=kn-1$$
		\end{itemize}
		\vspace{0.5cm}
		\item quindi: $$\frac{m}{m^*}=\frac{n-1}{kn-1}$$
		\item cos\'i che: \newline \\
			\hspace{8cm}$(<)$ poich\'e $\frac{1}{r}>1$
			$$\frac{m}{m^*}=\frac{n-1}{kn-1}\leq\frac{n-1}{\frac{n}{r}-1}<\frac{n-1}{\frac{n}{r}-\frac{1}{r}}=\frac{n-1}{\frac{1}{r}(n-1)}=r$$
	\end{itemize}
	\hfill$\square$
\end{flushleft}

% $$$$$$$$$$$$$$$$$$$$$$$$$$$$$$$$$$$$$$$$$$$$$$$$$$$$$$$$$$$$$$$$$$$$$$$$$$$$$$$$

\subsection*{miglioramento algoritmo: Greedy-Knapsack}
\addcontentsline{toc}{subsection}{miglioramento algoritmo: Greedy-Knapsack}
\begin{flushleft}
	\begin{itemize}
		\item osservazione:
		\begin{itemize}
			\item intuitivamente, Greedy-Knapsack non restituisce una buona approssimazione, poich\'e ignora l'oggetto avente il profitto massimo
		\end{itemize}
	\end{itemize}
\end{flushleft}

% $$$$$$$$$$$$$$$$$$$$$$$$$$$$$$$$$$$$$$$$$$$$$$$$$$$$$$$$$$$$$$$$$$$$$$$$$$$$$$$$

\subsection*{Greedy-Knapsack modificato}
\addcontentsline{toc}{subsection}{Greedy-Knapsack modificato}
\begin{flushleft}
	\begin{itemize}
		\item calcola una soluzione greedy $Q_{GR}$ e sia $m_{GR}$ la misura di quest'ultima
		\item considera l'oggetto $O_{\max}$ avente il massimo profitto $p_{\max}$ \item se $m_{GR}\geq p_{\max}$ restituisci $Q_{GR}$ altrimenti restituisci $Q=\{O_{\max}\}$
	\end{itemize}
\end{flushleft}

% $$$$$$$$$$$$$$$$$$$$$$$$$$$$$$$$$$$$$$$$$$$$$$$$$$$$$$$$$$$$$$$$$$$$$$$$$$$$$$$$

\subsection*{lemma 1: Greedy-Knapsack modificato}
\addcontentsline{toc}{subsection}{lemma 1: Greedy-Knapsack modificato}
\begin{flushleft}
	\begin{itemize}
		\item sia $o_j$ il primo oggetto che l'algoritmo Greedy-Knapsack non inserisce nel knapsack e sia:
			$$m_j=\sum_{i=1}^{j-1}p_i$$
		\item allora:
			$$m^*\leq m_j+p_j$$
	\end{itemize}
	\vspace{0.5cm}
	\textbf{dimostrazione:}
	\begin{itemize}
		\item $m^*\leq m_j+p_j$ deriva direttamente osservando semplicemente che, denotando con $v$ la somma dei volumi dei primi $j=1$ oggetti scelti, $m_j+p_j$ \'e il valore della soluzione ottima dell'istanza in cui il volume del knapsack \'e $v+a_j>b$
	\end{itemize}
\end{flushleft}

% $$$$$$$$$$$$$$$$$$$$$$$$$$$$$$$$$$$$$$$$$$$$$$$$$$$$$$$$$$$$$$$$$$$$$$$$$$$$$$$$

\subsection*{lemma 2: Greedy-Knapsack modificato}
\addcontentsline{toc}{subsection}{lemma 2: Greedy-Knapsack modificato}
\begin{flushleft}
	\begin{itemize}
		\item $m^*\leq m_{GR}+p_{\max}$
	\end{itemize}
	\vspace{0.5cm}
	\textbf{dimostrazione:}
	\begin{itemize}
		\item diretta conseguenza del procedente lemma osservando che $m_j\leq m_{GR}$ e $p_j\leq p_{\max}$, e quindi:
			$$m^*\leq m_j+p_j\leq m_{GR}+p_{\max}$$
		\item intuizione: l'algoritmo restituisce una soluzione di valore $\max\{m_{GR},p_{\max}\}$, che \'e almeno la met\'a di $m_{GR}+p_{\max}$, ovvero la met\'a di un upper bound di $m^*$
			$$\max\{m_{GR},p_{\max}\}\geq\frac{m_{GR}+p_{\max}}{2}$$
	\end{itemize}
\end{flushleft}


% $$$$$$$$$$$$$$$$$$$$$$$$$$$$$$$$$$$$$$$$$$$$$$$$$$$$$$$$$$$$$$$$$$$$$$$$$$$$$$$$

\subsection*{teorema: Greedy-Knapsack modificato \'e $\frac{1}{2}$-approssimante}
\addcontentsline{toc}{subsection}{teorema: Greedy-Knapsack modificato \'e $\frac{1}{2}$-approssimante}
\begin{flushleft}
	Greedy-Knapsack modificato \'e $\frac{1}{2}$-approssimante \newline \\
	\vspace{0.5cm}
	\textbf{dimostrazione:}
	\begin{itemize}
		\item $m_{Mod}\geq\max\{m_{GR},p_{\max}\}\geq\frac{(m_{GR}+p_{\max})}{2}\geq\frac{m^*}{2}$
	\end{itemize}
\end{flushleft}

% $$$$$$$$$$$$$$$$$$$$$$$$$$$$$$$$$$$$$$$$$$$$$$$$$$$$$$$$$$$$$$$$$$$$$$$$$$$$$$$$

\subsection*{problema: min multiprocessor scheduling}
\addcontentsline{toc}{subsection}{problema: min multiprocessor scheduling}
\begin{flushleft}
	\begin{itemize}
		\item INPUT:
		\begin{itemize}
			\item insieme di $n$ jobs $P$
			\item numero di processori $h$
			\item tempo di esecuzione $t_j$ $\forall p_j\in P$
		\end{itemize}
		\item SOLUZIONE:
		\begin{itemize}
			\item uno schedule per $P$, ovvero una funzione
				$$f:P\rightarrow\{1,\ldots,h\}$$
		\end{itemize}
		\item MISURA:
		\begin{itemize}
			\item $makespan$ o tempo di completamento di $f$, ovvero
				$$\max_{i\in[1,\ldots,h]}\sum_{p_j\in P\hspace{0.1cm}\vert\hspace{0.1cm}f(p_j)=i}t_j$$
		\end{itemize}
	\end{itemize}
\end{flushleft}

% $$$$$$$$$$$$$$$$$$$$$$$$$$$$$$$$$$$$$$$$$$$$$$$$$$$$$$$$$$$$$$$$$$$$$$$$$$$$$$$$

\subsection*{algoritmo: Greedy-Graham}
\addcontentsline{toc}{subsection}{algoritmo: Greedy-Graham}
\begin{flushleft}
	\begin{itemize}
		\item scelta greedy: ad ogni step assegna un job al processore meno carico
		\item $T_i(j)$:
		\begin{itemize}
			\item tempo di completamento (somma dei tempi di esecuzione dei jobs assegnati) del processore $i$ al termine del tempo $j$, ovvero una volta schedulati i primi $j$ jobs (in qualunque ordine)
		\end{itemize}
	\end{itemize}
	\begin{algorithm}
		\caption{Greedy-Graham}
		\begin{algorithmic}
			\STATE siano $p_1,\ldots,p_n$ i jobs elencati in un qualsiasi ordine
			\FOR{$j=1$ to $n$}
				\STATE assegna $p_j$ al processore $i$ avente il minimo $T_i(j-1)$ ovvero $f(p_j)i$
			\ENDFOR
			\RETURN lo schedule $i$
		\end{algorithmic}
	\end{algorithm}
	\begin{itemize}
		\item osservazione:
		\begin{itemize}
			\item se i jobs vengono schedulati in accordo con il tempo di arrivo, l'algoritmo assegna ciascun job senza conoscere quelli futuri, ovvero ONLINE
		\end{itemize}
	\end{itemize}
\end{flushleft}

% $$$$$$$$$$$$$$$$$$$$$$$$$$$$$$$$$$$$$$$$$$$$$$$$$$$$$$$$$$$$$$$$$$$$$$$$$$$$$$$$

\subsection*{teorema: Greedy-Graham \'e $\frac{2-1}{h}$-approssimante}
\addcontentsline{toc}{subsection}{teorema: Greedy-Graham \'e $\frac{2-1}{h}$-approssimante}
\begin{flushleft}
	l'algoritmo Greedy-Graham \'e $\frac{2-1}{h}$-approssimante, dove $h$ \'e il numero di processori \newline \\
	\vspace{0.5cm}
	\textbf{fatto:}
	\begin{itemize}
		\item dato $s\geq 0$ e $h$ numeri $a_1,\ldots,a_h\hspace{0.1cm}\vert\hspace{0.1cm}a_1+\ldots+a_h=s$, allora esiste $j$, $1\leq j\leq h$, tale che \newline
			$$a_j\geq\frac{s}{h}\hspace{3cm}a_1+\ldots+a_h<h\frac{s}{h}$$
		\item altrimenti, contraddizione
		\item analogamente, esiste $j'$, $1\leq j'\leq h$, tale che $a_{j'}\leq\frac{s}{h}$
		\item in altre parole, un numero \'e al massimo uguale alla media e uno maggiore o uguale alla media
		\item pertanto, $\min_j a_j\leq\frac{s}{h}$ e $\max_j a_j\geq\frac{s}{h}$
	\end{itemize}
	\vspace{0.5cm}
	\textbf{dimostrazione:}
\end{flushleft}

% $$$$$$$$$$$$$$$$$$$$$$$$$$$$$$$$$$$$$$$$$$$$$$$$$$$$$$$$$$$$$$$$$$$$$$$$$$$$$$$$

\subsection*{}
\addcontentsline{toc}{subsection}{}
\begin{flushleft}
\end{flushleft}

% $$$$$$$$$$$$$$$$$$$$$$$$$$$$$$$$$$$$$$$$$$$$$$$$$$$$$$$$$$$$$$$$$$$$$$$$$$$$$$$$

\subsection*{}
\addcontentsline{toc}{subsection}{}
\begin{flushleft}
\end{flushleft}

% $$$$$$$$$$$$$$$$$$$$$$$$$$$$$$$$$$$$$$$$$$$$$$$$$$$$$$$$$$$$$$$$$$$$$$$$$$$$$$$$

\subsection*{}
\addcontentsline{toc}{subsection}{}
\begin{flushleft}
\end{flushleft}

% $$$$$$$$$$$$$$$$$$$$$$$$$$$$$$$$$$$$$$$$$$$$$$$$$$$$$$$$$$$$$$$$$$$$$$$$$$$$$$$$

\subsection*{}
\addcontentsline{toc}{subsection}{}
\begin{flushleft}
\end{flushleft}

% $$$$$$$$$$$$$$$$$$$$$$$$$$$$$$$$$$$$$$$$$$$$$$$$$$$$$$$$$$$$$$$$$$$$$$$$$$$$$$$$

\subsection*{}
\addcontentsline{toc}{subsection}{}
\begin{flushleft}
\end{flushleft}

% $$$$$$$$$$$$$$$$$$$$$$$$$$$$$$$$$$$$$$$$$$$$$$$$$$$$$$$$$$$$$$$$$$$$$$$$$$$$$$$$

\subsection*{}
\addcontentsline{toc}{subsection}{}
\begin{flushleft}
\end{flushleft}

% $$$$$$$$$$$$$$$$$$$$$$$$$$$$$$$$$$$$$$$$$$$$$$$$$$$$$$$$$$$$$$$$$$$$$$$$$$$$$$$$

\subsection*{}
\addcontentsline{toc}{subsection}{}
\begin{flushleft}
\end{flushleft}

% $$$$$$$$$$$$$$$$$$$$$$$$$$$$$$$$$$$$$$$$$$$$$$$$$$$$$$$$$$$$$$$$$$$$$$$$$$$$$$$$

\subsection*{}
\addcontentsline{toc}{subsection}{}
\begin{flushleft}
\end{flushleft}

% $$$$$$$$$$$$$$$$$$$$$$$$$$$$$$$$$$$$$$$$$$$$$$$$$$$$$$$$$$$$$$$$$$$$$$$$$$$$$$$$

\subsection*{}
\addcontentsline{toc}{subsection}{}
\begin{flushleft}
\end{flushleft}

% $$$$$$$$$$$$$$$$$$$$$$$$$$$$$$$$$$$$$$$$$$$$$$$$$$$$$$$$$$$$$$$$$$$$$$$$$$$$$$$$

\subsection*{}
\addcontentsline{toc}{subsection}{}
\begin{flushleft}
\end{flushleft}

% $$$$$$$$$$$$$$$$$$$$$$$$$$$$$$$$$$$$$$$$$$$$$$$$$$$$$$$$$$$$$$$$$$$$$$$$$$$$$$$$

\subsection*{}
\addcontentsline{toc}{subsection}{}
\begin{flushleft}
\end{flushleft}

% $$$$$$$$$$$$$$$$$$$$$$$$$$$$$$$$$$$$$$$$$$$$$$$$$$$$$$$$$$$$$$$$$$$$$$$$$$$$$$$$

\subsection*{}
\addcontentsline{toc}{subsection}{}
\begin{flushleft}
\end{flushleft}

% $$$$$$$$$$$$$$$$$$$$$$$$$$$$$$$$$$$$$$$$$$$$$$$$$$$$$$$$$$$$$$$$$$$$$$$$$$$$$$$$

\subsection*{}
\addcontentsline{toc}{subsection}{}
\begin{flushleft}
\end{flushleft}

% $$$$$$$$$$$$$$$$$$$$$$$$$$$$$$$$$$$$$$$$$$$$$$$$$$$$$$$$$$$$$$$$$$$$$$$$$$$$$$$$

\subsection*{}
\addcontentsline{toc}{subsection}{}
\begin{flushleft}
\end{flushleft}

% $$$$$$$$$$$$$$$$$$$$$$$$$$$$$$$$$$$$$$$$$$$$$$$$$$$$$$$$$$$$$$$$$$$$$$$$$$$$$$$$

\subsection*{}
\addcontentsline{toc}{subsection}{}
\begin{flushleft}
\end{flushleft}

% $$$$$$$$$$$$$$$$$$$$$$$$$$$$$$$$$$$$$$$$$$$$$$$$$$$$$$$$$$$$$$$$$$$$$$$$$$$$$$$$

\newpage
