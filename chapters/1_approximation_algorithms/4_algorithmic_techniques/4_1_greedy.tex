\section*{algorithmic techniques: greedy}
\addcontentsline{toc}{section}{algorithmic techniques: greedy}

% $$$$$$$$$$$$$$$$$$$$$$$$$$$$$$$$$$$$$$$$$$$$$$$$$$$$$$$$$$$$$$$$$$$$$$$$$$$$$$$$

\subsection*{caratteristiche}
\addcontentsline{toc}{subsection}{caratteristiche}
\begin{flushleft}
	\begin{itemize}
		\item la soluzione viene determinata in step
		\item ad ogni step l'algoritmo esegue la scelta che sembra essere la migliore in quello step, senza considerare le possibili conseguenze nei futuri step
	\end{itemize}
\end{flushleft}

% $$$$$$$$$$$$$$$$$$$$$$$$$$$$$$$$$$$$$$$$$$$$$$$$$$$$$$$$$$$$$$$$$$$$$$$$$$$$$$$$

\subsection*{problema: max 0-1 knapsack}
\addcontentsline{toc}{subsection}{problema: max 0-1 knapsack}
\begin{flushleft}
	\begin{itemize}
		\item INPUT:
		\begin{itemize}
			\item un insieme finito di oggetti $O$
			\item un profitto intero $p_i$, $\forall o_i\in O$
			\item un volume intero $a_i$, $\forall o_i\in O$
			\item un intero positivo $b$ ($b>0$)
		\end{itemize}
		\item SOLUZIONE:
		\begin{itemize}
			\item un sottoinsieme di oggetti $Q\subseteq O$ tale che $\sum_{o_i\in Q}a_i\leq b$
		\end{itemize}
		\item MISURA:
		\begin{itemize}
			\item profitto totale degli oggetti scelti, ovvero $\sum_{o_i\in Q}p_i$
		\end{itemize}
		\vspace{0.5cm}
		\item senza perdere di generalit\'a, in seguito, assumeremo sempre che:
		\begin{itemize}
			\item $a_i\leq b$, $\forall o_i\in O$
			\item $p_i>0$, $\forall o_i\in O$
		\end{itemize}
	\end{itemize}
\end{flushleft}

% $$$$$$$$$$$$$$$$$$$$$$$$$$$$$$$$$$$$$$$$$$$$$$$$$$$$$$$$$$$$$$$$$$$$$$$$$$$$$$$$

\subsection*{max 0-1 knapsack: descrizione della scelta greedy}
\addcontentsline{toc}{subsection}{max 0-1 knapsack: descrizione della scelta greedy}
\begin{flushleft}
	\begin{itemize}
		\item nella scelta greedy:
		\begin{itemize}
			\item non possiamo considerare solo il profitto degli oggetti, in quanto il loro volume potrebbe essere troppo grande
			\item non possiamo considerare solo il volume degli oggetti, in quanto il loro profitto potrebbe essere troppo basso
		\end{itemize}
		\item idea: consideriamo gli oggetti in base al profitto per unit\'a di volume, ovvero in base al rapporto $$\frac{p_i}{a_i}\text{, }\forall o_i\in O$$
		\item l'algoritmo greedy seleziona gli oggetti in ordine decrescente di profitto per volume
	\end{itemize}
\end{flushleft}

% $$$$$$$$$$$$$$$$$$$$$$$$$$$$$$$$$$$$$$$$$$$$$$$$$$$$$$$$$$$$$$$$$$$$$$$$$$$$$$$$

\newpage
\subsection*{algoritmo: Greedy-Knapsack}
\addcontentsline{toc}{subsection}{algoritmo: Greedy-Knapsack}
\begin{flushleft}
	\begin{algorithm}
		\caption{Greedy-Knapsack}
		\begin{algorithmic}
			\STATE \color{gray} // Q = insieme degli oggetti scelti \color{black}
			\STATE $Q=\emptyset$
			\STATE \color{gray} // v = volume del sottoinsieme corrente degli oggetti scelti \color{black}
			\STATE $v=0$
			\STATE ordina gli oggetti in ordine decrescente di profitto per volume $\frac{p_i}{a_i}$
			\STATE siano $o_1,\ldots,o_n$ gli oggetti elencati secondo tale ordine
			\FOR{$i=1$ to $n$}
				\IF{$v+a_i\leq b$}
					\STATE $Q=Q\cup\{o_i\}$
					\STATE $v=v+a_i$
				\ENDIF
			\ENDFOR
			\RETURN $Q$
		\end{algorithmic}
	\end{algorithm}
\end{flushleft}

% $$$$$$$$$$$$$$$$$$$$$$$$$$$$$$$$$$$$$$$$$$$$$$$$$$$$$$$$$$$$$$$$$$$$$$$$$$$$$$$$

\subsection*{teorema: $\forall r<1$ Greedy-Knapsack non \'e r-approssimante}
\addcontentsline{toc}{subsection}{teorema: $\forall r<1$ Greedy-Knapsack non \'e r-approssimante}
\begin{flushleft}
	$\forall r<1$ dato, Greedy-Knapsack non \'e r-approssimante \newline \\
	\textbf{dimostrazione:}
	\begin{itemize}
		\item dato un intero $k=\lceil\frac{1}{r}\rceil$, consideriamo la seguente istanza di max 0-1 knapsack
		\item $\forall n\geq 2$
		\begin{itemize}
			\item $b=kn$ \'e il volume del knapsack
			\item $n-1$ oggetti con profitto $p_i=1$ e volume $a_i=1$
			\item $1$ oggetto con profitto $b-1$ e volume $b$
		\end{itemize}
		\item soluzione restituita:
		\begin{itemize}
			\item l'insieme dei primi $n-1$ oggetti, ovvero $m=n-1$
		\end{itemize}
		\item soluzione ottima
		\begin{itemize}
			\item l'insieme contenente solo l'$n$-esimo oggetto, ovvero $$m^*=b-1=kn-1$$
		\end{itemize}
		\vspace{0.5cm}
		\item quindi: $$\frac{m}{m^*}=\frac{n-1}{kn-1}$$
		\item cos\'i che: \newline \\
			\hspace{8cm}$(<)$ poich\'e $\frac{1}{r}>1$
			$$\frac{m}{m^*}=\frac{n-1}{kn-1}\leq\frac{n-1}{\frac{n}{r}-1}<\frac{n-1}{\frac{n}{r}-\frac{1}{r}}=\frac{n-1}{\frac{1}{r}(n-1)}=r$$
		\item \color{gray} $\forall r<1\rightarrow\frac{m}{m^*}<r$, invece di $\forall r\leq 1\rightarrow\frac{m}{m^*}\geq r$
	\end{itemize}
	\hfill$\square$
\end{flushleft}

% $$$$$$$$$$$$$$$$$$$$$$$$$$$$$$$$$$$$$$$$$$$$$$$$$$$$$$$$$$$$$$$$$$$$$$$$$$$$$$$$

\subsection*{miglioramento algoritmo: Greedy-Knapsack}
\addcontentsline{toc}{subsection}{miglioramento algoritmo: Greedy-Knapsack}
\begin{flushleft}
	\begin{itemize}
		\item osservazione:
		\begin{itemize}
			\item intuitivamente, Greedy-Knapsack non restituisce una buona approssimazione, poich\'e ignora l'oggetto avente il profitto massimo
		\end{itemize}
	\end{itemize}
\end{flushleft}

% $$$$$$$$$$$$$$$$$$$$$$$$$$$$$$$$$$$$$$$$$$$$$$$$$$$$$$$$$$$$$$$$$$$$$$$$$$$$$$$$

\subsection*{Greedy-Knapsack modificato}
\addcontentsline{toc}{subsection}{Greedy-Knapsack modificato}
\begin{flushleft}
	\begin{itemize}
		\item calcola una soluzione greedy $Q_{GR}$ e sia $m_{GR}$ la misura di quest'ultima
		\item considera l'oggetto $O_{\max}$ avente il massimo profitto $p_{\max}$ \item se $m_{GR}\geq p_{\max}$ restituisci $Q_{GR}$ altrimenti restituisci $Q=\{O_{\max}\}$
	\end{itemize}
\end{flushleft}

% $$$$$$$$$$$$$$$$$$$$$$$$$$$$$$$$$$$$$$$$$$$$$$$$$$$$$$$$$$$$$$$$$$$$$$$$$$$$$$$$

\subsection*{lemma 1: Greedy-Knapsack modificato $m^*\leq m_j+p_j$}
\addcontentsline{toc}{subsection}{lemma 1: Greedy-Knapsack modificato $m^*\leq m_j+p_j$}
\begin{flushleft}
	\begin{itemize}
		\item sia $o_j$ il primo oggetto che l'algoritmo Greedy-Knapsack non inserisce nel knapsack e sia:
			$$m_j=\sum_{i=1}^{j-1}p_i$$
		\item allora:
			$$m^*\leq m_j+p_j$$
	\end{itemize}
	\vspace{0.5cm}
	\textbf{dimostrazione:}
	\begin{itemize}
		\item $m^*\leq m_j+p_j$ deriva direttamente osservando semplicemente che, denotando con $v$ la somma dei volumi dei primi $j-1$ oggetti scelti, $m_j+p_j$ \'e il valore della soluzione ottima dell'istanza in cui il volume del knapsack \'e $v+a_j>b$
	\end{itemize}
	\hfill$\square$
\end{flushleft}

% $$$$$$$$$$$$$$$$$$$$$$$$$$$$$$$$$$$$$$$$$$$$$$$$$$$$$$$$$$$$$$$$$$$$$$$$$$$$$$$$

\subsection*{lemma 2: Greedy-Knapsack modificato $m^*\leq m_{GR}+p_{\max}$}
\addcontentsline{toc}{subsection}{lemma 2: Greedy-Knapsack modificato $m^*\leq m_{GR}+p_{\max}$}
\begin{flushleft}
	\begin{itemize}
		\item[] $m^*\leq m_{GR}+p_{\max}$
	\end{itemize}
	\vspace{0.5cm}
	\textbf{dimostrazione:}
	\begin{itemize}
		\item diretta conseguenza del procedente lemma osservando che $m_j\leq m_{GR}$ e $p_j\leq p_{\max}$, e quindi:
			$$m^*\leq m_j+p_j\leq m_{GR}+p_{\max}$$
		\item intuizione: l'algoritmo restituisce una soluzione di valore $\max\{m_{GR},p_{\max}\}$, che \'e almeno la met\'a di $m_{GR}+p_{\max}$, ovvero la met\'a di un upper bound di $m^*$
			$$\max\{m_{GR},p_{\max}\}\geq\frac{m_{GR}+p_{\max}}{2}$$
	\end{itemize}
	\hfill$\square$
\end{flushleft}


% $$$$$$$$$$$$$$$$$$$$$$$$$$$$$$$$$$$$$$$$$$$$$$$$$$$$$$$$$$$$$$$$$$$$$$$$$$$$$$$$

\subsection*{teorema: Greedy-Knapsack modificato \'e $\frac{1}{2}$-approssimante}
\addcontentsline{toc}{subsection}{teorema: Greedy-Knapsack modificato \'e $\frac{1}{2}$-approssimante}
\begin{flushleft}
	Greedy-Knapsack modificato \'e $\frac{1}{2}$-approssimante \newline \\
	\vspace{0.5cm}
	\textbf{dimostrazione:}
	\begin{itemize}
		\item $m_{Mod}\geq\max\{m_{GR},p_{\max}\}\geq\frac{(m_{GR}+p_{\max})}{2}\geq\frac{m^*}{2}$
	\end{itemize}
	\hfill$\square$
\end{flushleft}

% $$$$$$$$$$$$$$$$$$$$$$$$$$$$$$$$$$$$$$$$$$$$$$$$$$$$$$$$$$$$$$$$$$$$$$$$$$$$$$$$

\subsection*{problema: min multiprocessor scheduling}
\addcontentsline{toc}{subsection}{problema: min multiprocessor scheduling}
\begin{flushleft}
	\begin{itemize}
		\item INPUT:
		\begin{itemize}
			\item insieme di $n$ jobs $P$
			\item numero di processori $h$
			\item tempo di esecuzione $t_j$, $\forall p_j\in P$
		\end{itemize}
		\item SOLUZIONE:
		\begin{itemize}
			\item uno schedule per $P$, ovvero una funzione
				$$f:P\rightarrow\{1,\ldots,h\}$$
		\end{itemize}
		\item MISURA:
		\begin{itemize}
			\item $makespan$ o tempo di completamento di $f$, ovvero
				$$\max_{i\in[1,\ldots,h]}\sum_{p_j\in P\hspace{0.1cm}\vert\hspace{0.1cm}f(p_j)=i}t_j$$
		\end{itemize}
	\end{itemize}
\end{flushleft}

% $$$$$$$$$$$$$$$$$$$$$$$$$$$$$$$$$$$$$$$$$$$$$$$$$$$$$$$$$$$$$$$$$$$$$$$$$$$$$$$$

\subsection*{algoritmo: Greedy-Graham}
\addcontentsline{toc}{subsection}{algoritmo: Greedy-Graham}
\begin{flushleft}
	\begin{itemize}
		\item scelta greedy: ad ogni step assegna un job al processore meno carico
		\item $T_i(j)$:
		\begin{itemize}
			\item tempo di completamento (somma dei tempi di esecuzione dei jobs assegnati) del processore $i$ al termine del tempo $j$, ovvero una volta schedulati i primi $j$ jobs (in qualunque ordine)
		\end{itemize}
	\end{itemize}
	\begin{algorithm}
		\caption{Greedy-Graham}
		\begin{algorithmic}
			\STATE siano $p_1,\ldots,p_n$ i jobs elencati in un qualsiasi ordine
			\FOR{$j=1$ to $n$}
				\STATE assegna $p_j$ al processore $i$ avente il minimo $T_i(j-1)$ ovvero $f(p_j)=i$
			\ENDFOR
			\RETURN schedule $i$
		\end{algorithmic}
	\end{algorithm}
	\begin{itemize}
		\item osservazione:
		\begin{itemize}
			\item se i jobs vengono schedulati in accordo con il tempo di arrivo, l'algoritmo assegna ciascun job senza conoscere quelli futuri, ovvero ONLINE
		\end{itemize}
	\end{itemize}
\end{flushleft}

% $$$$$$$$$$$$$$$$$$$$$$$$$$$$$$$$$$$$$$$$$$$$$$$$$$$$$$$$$$$$$$$$$$$$$$$$$$$$$$$$

\subsection*{teorema: Greedy-Graham \'e $2-\frac{1}{h}$-approssimante}
\addcontentsline{toc}{subsection}{teorema: Greedy-Graham \'e $2-\frac{1}{h}$-approssimante}
\begin{flushleft}
	l'algoritmo Greedy-Graham \'e $2-\frac{1}{h}$-approssimante, dove $h$ \'e il numero di processori \newline \\
	\vspace{0.5cm}
	\textbf{fatto:}
	\begin{itemize}
		\item dato $s\geq 0$ e $h$ numeri $a_1,\ldots,a_h\hspace{0.1cm}\vert\hspace{0.1cm}a_1+\ldots+a_h=s$, allora esiste $j$, $1\leq j\leq h$, tale che \newline
			$$a_j\geq\frac{s}{h}$$
		\item altrimenti, contraddizione ($a_1+\ldots+a_h<h\frac{s}{h}=s$)
		\item analogamente, esiste $j'$, $1\leq j'\leq h$, tale che $a_{j'}\leq\frac{s}{h}$
		\item in altre parole, un numero \'e al massimo uguale alla media e uno maggiore o uguale alla media
		\item pertanto, $\min_j a_j\leq\frac{s}{h}$ e $\max_j a_j\geq\frac{s}{h}$
	\end{itemize}
	\vspace{0.5cm}
	\textbf{dimostrazione:}
	\begin{itemize}
		\item sia $T$ la somma di tutti i tempi di esecuzione dei job, ovvero
			$$T=\sum_{j=1}^n t_j$$
		\item siano $T_1^*,T_2^*,\ldots,T_h^*$ i tempi di completamento degli $h$ processori nella soluzione ottima
		\item poich\'e $T_1^*+T_2^*+\ldots+T_h^*=T$ dal precedente 'fatto', esiste $j$ tale che $T_j^*\geq\frac{T}{h}$
		\item quindi:
			$$m^*\geq T_j^*\geq\frac{T}{h}$$
		\item sia $k$ il processore con il massimo tempo di completamento nello schedule $f$ restituito dall'algoritmo, ovvero con $T_k(n)$ massimo
		\item in pi\'u sia $p_l$ l'ultimo job assegnato al processore $k$
		\item dato che, per la scelta greedy, $p_l$ \'e stato assegnato ad uno dei processori meno carichi all'inizio dello step $l$, sempre per il 'fatto' precedente, abbiamo:
			$$T_k(l-1)\leq\frac{\sum_{j<l}t_j}{h}\leq\frac{T-t_l}{h}$$
		\item dato che la somma dei tempi di esecuzione di tutti i jobs assegnati prima di $p_l$ \'e al massimo ($\leq$) $T-t_l$
		\item pertanto:
			$$m=T_k(n)=T_k(l-1)+t_l\leq\frac{T-t_l}{h}+t_l=$$
			$$=\color{gray}\frac{T-t_l+ht_l}{h}=\frac{T}{h}-\frac{1+h}{h}t_l\color{black}=\frac{T}{h}+\frac{h-1}{h}t_l\leq\ldots$$
		\item poich\'e $\frac{T}{h}\leq m^*$ e $t_l\leq m^*$
			$$\ldots\leq m^*+\frac{h-1}{h}m^*\color{gray}=\frac{hm^*+(h-1)m^*}{h}=\frac{hm^*+hm^*-m^*}{h}=$$
			$$\color{gray}=\frac{2hm^*-m^*}{h}=\frac{2h-1}{h}m^*\color{black}=(2-\frac{1}{h})m^*$$
		\item e quindi:
			$$\frac{m}{m^*}\leq 2-\frac{1}{h}$$
	\end{itemize}
	\hfill$\square$
	\begin{itemize}
		\item osservazioni:
		\begin{itemize}
			\item quando $h$ cresce, il rapporto di approssimazione $2-\frac{1}{h}$ tende a $2$
			\item l'analisi \'e stretta, ovvero vale il seguente teorema
		\end{itemize}
	\end{itemize}
\end{flushleft}

% $$$$$$$$$$$$$$$$$$$$$$$$$$$$$$$$$$$$$$$$$$$$$$$$$$$$$$$$$$$$$$$$$$$$$$$$$$$$$$$$

\subsection*{teorema: Greedy-Graham non \'e $r$-approssimante per $r<2-\frac{1}{h}$}
\addcontentsline{toc}{subsection}{teorema: Greedy-Graham non \'e $r$-approssimante per $r<2-\frac{1}{h}$}
\begin{flushleft}
	Greedy-Graham non \'e $r$-approssimante per $r<2-\frac{1}{h}$ \newline \\
	\vspace{0.5cm}
	\textbf{dimostrazione:}
	\begin{itemize}
		\item considera la seguente istanza:
		\begin{itemize}
			\item $h(h-1)$ jobs con tempo di esecuzione $1$
			\item $1$ job con tempo di esecuzione $h$
		\end{itemize}
		\item Greedy-Graham assegna i jobs nella seguente maniera:
			% TODO: grafico %
		\item e quindi: $$m=h+h-1=2h-1$$
		\item la soluzione ottima pu\'o essere ottenuta assegnando il job pi\'u lungo ad un processore e distribuendo ugualmente i jobs pi\'u corti tra i processori restanti:
			% TODO: grafico %
		\item e quindi: $$m^*=h$$
		\item in conclusione:
			$$\frac{m}{m^*}=\frac{2h-1}{h}=2-\frac{1}{h}\hspace{2cm}\text{(diverso da }\leq 2-\frac{1}{h}\text{)}$$
	\end{itemize}
	\hfill$\square$
\end{flushleft}

% $$$$$$$$$$$$$$$$$$$$$$$$$$$$$$$$$$$$$$$$$$$$$$$$$$$$$$$$$$$$$$$$$$$$$$$$$$$$$$$$

\subsection*{migliorare il rapporto di approssimazione $r$ per Greedy-Graham}
\addcontentsline{toc}{subsection}{migliorare il rapporto di approssimazione $r$ per Greedy-Graham}
\begin{flushleft}
	\begin{itemize}
		\item DOMANDA: come possiamo migliorare il rapporto di approssimazione $r$
		\item richiamiamo rapidamente gli step base della dimostrazione del rapporto di approssimazione di Greedy-Graham
		\item abbiamo utilizzato i seguenti $lower\text{ }bounds$ per il valore della soluzione ottima:
		\begin{itemize}
			\item $m^*\geq\frac{T}{h}$, come in qualsiasi soluzione almeno 1 processore deve avere tempo di completamento $\frac{T}{h}$ (richiamiamo che $T=\sum_j t_j$)
			\item $m^*\geq t_j$, per ogni job $p_j$, come in qualsiasi soluzione uno dei processori deve eseguire $p_j$
		\end{itemize}
		\item abbiamo utilizzato il seguente $upper\text{ }bound$ per il valore della soluzione restituita:
		\begin{itemize}
			\item per limitare superiormente il valore della soluzione restituita, se $k$ \'e uno dei processori pi\'u carichi e $p_l$ \'e l'ultimo job assegnato a $k$, per la scelta greedy:
				$$T_k(l-1)\leq\frac{\sum_{j<l}t_j}{h}\leq\frac{T-t_l}{h}$$
		\end{itemize}
		\item quindi possiamo derivare la seguente disuguaglianza:
			$$m=T_k(n)=T_k(l-1)+t_l\leq\frac{T-t_l}{h}+t_l=$$
			$$=\color{gray}\frac{T-t_l+ht_l}{h}=\frac{T}{h}-\frac{1+h}{h}t_l\color{black}=\frac{T}{h}+\frac{h-1}{h}t_l\leq\ldots$$
		\item poich\'e $\frac{T}{h}\leq m^*$ e $t_l\leq m^*$
			$$\ldots\leq m^*+\frac{h-1}{h}m^*\color{gray}=\frac{hm^*+(h-1)m^*}{h}=\frac{hm^*+hm^*-m^*}{h}=$$
			$$\color{gray}=\frac{2hm^*-m^*}{h}=\frac{2h-1}{h}m^*\color{black}=(2-\frac{1}{h})m^*$$
		\item idea per il miglioramento: decrementa $t_l$ il pi\'u possibile e trova un rapporto di approssimazione migliore sfruttando le disuguaglianze
			$$m\leq\frac{T}{h}+\frac{h-1}{h}t_l\leq m^*+\frac{h-1}{h}t_l$$
		\item modificando l'algoritmo e/o migliorando l'analisi vedremo come limitare superiormente $t_l$ progressivamente con:
		\begin{itemize}
			\item $\frac{m^*}{2}$ ($\frac{3}{2}$-approssimante),
			\item $\frac{m^*}{3}$ ($\frac{4}{3}$-approssimante),
			\item e arbitrariamente piccolo, ovvero $\epsilon m^*$ ($(1+\epsilon)$-approssimante), cio\'e un PTAS 
		\end{itemize}
	\end{itemize}
\end{flushleft}

% $$$$$$$$$$$$$$$$$$$$$$$$$$$$$$$$$$$$$$$$$$$$$$$$$$$$$$$$$$$$$$$$$$$$$$$$$$$$$$$$

\subsection*{Greedy-Graham, primo miglioramento}
\addcontentsline{toc}{subsection}{Greedy-Graham, primo miglioramento}
\begin{flushleft}
	\begin{itemize}
		\item assegnare i jobs dal pi\'u lungo al pi\'u corto
		\item ci\'o ci consente di evitare il caso peggiore dell'algoritmo di Graham, ovvero il fatto che un job lungo arrivi alla fine, sbilanciando significativamente il carico dei processori
	\end{itemize}
\end{flushleft}

% $$$$$$$$$$$$$$$$$$$$$$$$$$$$$$$$$$$$$$$$$$$$$$$$$$$$$$$$$$$$$$$$$$$$$$$$$$$$$$$$

\subsection*{algoritmo: Ordered-Greedy}
\addcontentsline{toc}{subsection}{algoritmo: Ordered-Greedy}
\begin{flushleft}
	\begin{algorithm}
		\caption{Ordered-Greedy}
		\begin{algorithmic}
			\STATE siano $p_1,p_2,\ldots,p_n$ i job elencati in ordine decrescente di tempo di
			esecuzione, ovvero tale che $t_1\geq t_2\geq\ldots\geq t_n$
			\FOR{$j=1$ to $n$}
				\STATE assegna $p_j$ al processore $i$ con il minimo $T_i(j-1)$, ovvero $f(p_j)=i$
			\ENDFOR
			\RETURN schedule $f$
		\end{algorithmic}
	\end{algorithm}
	\begin{itemize}
		\item vediamo un'analisi pi\'u semplice che porta ad un rapporto di approssimazione di circa $\frac{3}{2}$
	\end{itemize}
\end{flushleft}

% $$$$$$$$$$$$$$$$$$$$$$$$$$$$$$$$$$$$$$$$$$$$$$$$$$$$$$$$$$$$$$$$$$$$$$$$$$$$$$$$

\subsection*{lemma: Ordered-Greedy $t_{h+1}\leq\frac{m^*}{2}$}
\addcontentsline{toc}{subsection}{lemma: Ordered-Greedy $t_{h+1}\leq\frac{m^*}{2}$}
\begin{flushleft}
	se $n>h$, allora $t_{h+1}\leq\frac{m^*}{2}$ \newline \\
	\vspace{0.5cm}
	\textbf{dimostrazione:}
	\begin{itemize}
		\item dall'ordinamento dei jobs, i primi $h+1$ hanno tutti un tempo di esecuzione $\geq t_{h+1}$
		\item ma allora $m^*\geq 2 t_{h+1}$, poich\'e in ogni schedule almeno $1$ degli $h$ processori deve ricevere almeno $2$ dei primi $h+1$ job
	\end{itemize}
	\hfill$\square$
\end{flushleft}

% $$$$$$$$$$$$$$$$$$$$$$$$$$$$$$$$$$$$$$$$$$$$$$$$$$$$$$$$$$$$$$$$$$$$$$$$$$$$$$$$

\subsection*{teorema: Ordered-Greedy \'e $(\frac{3}{2}-\frac{1}{2h})$-approssimante}
\addcontentsline{toc}{subsection}{teorema: Ordered-Greedy \'e $(\frac{3}{2}-\frac{1}{2h})$-approssimante}
\begin{flushleft}
	Ordered-Greedy \'e $(\frac{3}{2}-\frac{1}{2h})$-approssimante \newline \\
	\vspace{0.5cm}
	\textbf{dimostrazione:}
	\begin{itemize}
		\item di nuovo sia $k$ uno dei processori pi\'u carichi (alla fine)
		\item se $k$ ha $1$ solo job, allora chiaramente la soluzione ritornata \'e ottima
		\item altrimenti considera l'ultimo job $p_l$ assegnato a $k$
		\item dato che $p_l$ non \'e il primo job assegnato a $k$, $l\geq h+1$ e quindi $t_l\leq t_{h+1}\leq\frac{m^*}{2}$, e cos\'i:
			$$m\leq\frac{T}{h}+\frac{h-1}{h}t_l\leq m^*+\frac{h-1}{h}\frac{m^*}{2}\color{gray}=$$
			$$\color{gray}=m^*+\frac{m^*(h-1)}{2h}=\frac{2hm^*+m^*(h-1)}{2h}=\frac{2hm^*+hm^*-m^*}{2h}=$$
			$$\color{gray}=\frac{3hm^*-m^*}{2h}=(\frac{3h-1}{2h})m^*=(\frac{3h}{2h}-\frac{1}{2h})m^*\color{black}=(\frac{3}{2}-\frac{1}{2h})m^*$$
		\item quindi:
			$$\frac{m}{m^*}\leq\frac{3}{2}-\frac{1}{2h}$$
	\end{itemize}
	\hfill$\square$
\end{flushleft}

% $$$$$$$$$$$$$$$$$$$$$$$$$$$$$$$$$$$$$$$$$$$$$$$$$$$$$$$$$$$$$$$$$$$$$$$$$$$$$$$$

\subsection*{problema: max cut}
\addcontentsline{toc}{subsection}{problema: max cut}
\begin{flushleft}
	\begin{itemize}
		\item INPUT: grafo $G=(V,E)$
		\item SOLUZIONE: una partizione di $V$ in 2 sottoinsiemi $V_1$ e $V_2$, ovvero tale che:
			$$V_1\cup V_2=V\text{ e }V_1\cap V_2=\emptyset$$
		\item MISURA: la cardinalit\'a del taglio, ovvero il numero di archi con un estremo (nodo) in $V_1$ e un estremo in $V_2$, cio\'e:
			$$\vert\{\{u,v\}\hspace{0.1cm}\vert\hspace{0.1cm}(u\in V_1\land v\in V_2)\lor(u\in V_2\land v\in V_1)\}\vert$$
	\end{itemize}
\end{flushleft}

% $$$$$$$$$$$$$$$$$$$$$$$$$$$$$$$$$$$$$$$$$$$$$$$$$$$$$$$$$$$$$$$$$$$$$$$$$$$$$$$$

\subsection*{algoritmo: Greedy-Max-Cut}
\addcontentsline{toc}{subsection}{algoritmo: Greedy-Max-Cut}
\begin{flushleft}
	\begin{algorithm}
		\caption{Greedy-Max-Cut}
		\begin{algorithmic}
			\STATE $V_1=V_2=\emptyset$
			\FOR{$i=1$ to $n$}
				\STATE \color{gray} // $\Delta_i$ = set di archi tra $i$ e i nodi $j<i$ (adiacenti) \color{black}
				\STATE $\Delta_i=\{\{i,j\}\in E\text{ }\vert\text{ }j<i\}$
				\STATE \color{gray} // $U_i$ = set di nodi gi\'a inseriti (adiacenti ad $i$, all'inizio dello step $i$) \color{black}
				\STATE $U_i=\{j\text{ }\vert\text{ }\{i,j\}\in\Delta_i\}$
				\STATE $\delta_i=\vert\Delta_i\vert=\vert U_i\vert$
				\STATE $\delta_{1i}=\vert V_1\cap U_i\vert$
				\STATE $\delta_{2i}=\vert V_2\cap U_i\vert$
				\STATE \color{gray} // chiaramente $\delta_{1i}+\delta_{2i}=\delta_i$ \color{black}
				\IF{$\delta_{1i}>\delta_{2i}$}
					\STATE $V_2=V_2\cup\{i\}$
				\ELSE
					\STATE $V_1=V_1\cup\{i\}$
				\ENDIF
			\ENDFOR
			\RETURN $V_1,V_2$
		\end{algorithmic}
	\end{algorithm}
	\begin{itemize}
		\item per semplicit\'a sia $V=\{1,\ldots,n\}$
		\item l'algoritmo ad ogni step inserisce un nuovo nodo in $V_1$ o in $V_2$
		\item scelta greedy:
		\begin{itemize}
			\item allo step $i$, il nodo $i$ viene inserito in modo da massimizzare il numero di archi nuovi nel taglio, ovvero in $V_1$ se il numero di archi che ha verso i nodi gi\'a inseriti in $V_2$ \'e maggiore ($\geq$) del numero di archi che ha verso quelli in $V_1$, altrimenti in $V_2$ ($<$)
		\end{itemize}
	\end{itemize}
\end{flushleft}

% $$$$$$$$$$$$$$$$$$$$$$$$$$$$$$$$$$$$$$$$$$$$$$$$$$$$$$$$$$$$$$$$$$$$$$$$$$$$$$$$

\subsection*{teorema: Greedy-Max-Cut \'e $\frac{1}{2}$-approssimante}
\addcontentsline{toc}{subsection}{teorema: Greedy-Max-Cut \'e $\frac{1}{2}$-approssimante}
\begin{flushleft}
	Greedy-Max-Cut \'e $\frac{1}{2}$-approssimante \newline \\
	\vspace{0.5cm}
	\textbf{dimostrazione:}
	\begin{itemize}
		\item chiaramente poich\'e quel taglio pu\'o solo contenere un sottoinsieme di tutti gli archi in $E$
			$$m^*\leq\vert E\vert$$
		\item mostriamo ora che la misura $m$ del taglio restituita dall'algoritmo \'e almeno la met\'a del numero totale di archi, ovvero:
			$$m\geq\frac{\vert E\vert}{2}$$
		\item ci\'o implica chiaramente l'affermazione, poich\'e
			$$\frac{m}{m^*}\geq\frac{\frac{\vert E\vert}{2}}{\vert E\vert}=\frac{1}{2}$$
		\item poich\'e gli insiemi $\Delta_i$ determinati dall'algoritmo formano una partizione di $E$ e per definizione $\delta_i=\vert\Delta_i\vert$:
			$$\sum_{i=1}^n\delta_i=\sum_{i=1}^n\vert\Delta_i\vert=\vert E\vert$$
		\item inoltre, il numero di archi aggiunti al taglio durante lo step $i$, ovvero con un estremo in $V_1$ e l'altro in $V_2$ (dopo l'esecuzione dell' $i$-esima iterazione dell'istruzione $for$), \'e:
			$$\max(\delta_{1i},\delta_{2i})\geq\frac{(\delta_{1i}+\delta_{2i})}{2}=\frac{\delta_i}{2}$$
		\item quindi:
			$$m=\sum_{i=1}^n\max(\delta_{1i},\delta_{2i})\geq\sum_{i=1}^n\frac{\delta_i}{2}=\frac{\vert E\vert}{2}$$
	\end{itemize}
	\hfill$\square$
\end{flushleft}

% $$$$$$$$$$$$$$$$$$$$$$$$$$$$$$$$$$$$$$$$$$$$$$$$$$$$$$$$$$$$$$$$$$$$$$$$$$$$$$$$

\subsection*{conclusioni sulla tecnica greedy}
\addcontentsline{toc}{subsection}{conclusioni sulla tecnica greedy}
\begin{flushleft}
	\begin{itemize}
		\item tutti gli algoritmi visti fin ora hanno complessit\'a temporale polinomiale
		\item gli algoritmi greedy hanno buone performance in pratica poich\'e possono essere implementati in modo semplice
		\item ma come abbiamo visto, compiere la scelta che sembra migliore a ciascun singolo step, senza badare alle conseguenze future, in generale non permette di trovare la soluzione ottima
	\end{itemize}
\end{flushleft}

% $$$$$$$$$$$$$$$$$$$$$$$$$$$$$$$$$$$$$$$$$$$$$$$$$$$$$$$$$$$$$$$$$$$$$$$$$$$$$$$$

\newpage
