\section*{computational complexity}
\addcontentsline{toc}{section}{computational complexity}

% $$$$$$$$$$$$$$$$$$$$$$$$$$$$$$$$$$$$$$$$$$$$$$$$$$$$$$$$$$$$$$$$$$$$$$$$$$$$$$$$

\subsection*{def: problema in computer science}
\addcontentsline{toc}{subsection}{def: problema in computer science}
un problema $\pi$ \'e una relazione
$$\pi\subseteq I_{\pi}\times S_{\pi}$$
dove:
\begin{itemize}
	\item $I_{\pi}=$ insieme delle istanze di input del problema
	\item $S_{\pi}=$ insieme delle soluzioni del problema
\end{itemize}

% $$$$$$$$$$$$$$$$$$$$$$$$$$$$$$$$$$$$$$$$$$$$$$$$$$$$$$$$$$$$$$$$$$$$$$$$$$$$$$$$

\subsection*{tipologie di problema}
\addcontentsline{toc}{subsection}{tipologie di problema}
\begin{itemize}
	\item decisione
		\begin{itemize}
			\item si verifica se una data propriet\'a \'e valida per un determinato input
			\item $S_{\pi}=\{true,false\}$ o semplicemente $S_{\pi}=\{0,1\}$ e la relazione $\pi\subseteq I_{\pi}\times S_{\pi}$ corrisponde ad una funzione $$f:I_{\pi}\rightarrow\{0,1\}$$ 
			\item esempi: soddisfacibilit\'a, test di connettivit\'a di un grafo, etc....
		\end{itemize}
	\item ricerca
	\begin{itemize}
		\item data un'istanza $x\in I_{\pi}$, si chiede di determinare una soluzione $y\in S_{\pi}$ tale che la coppia $(x,y)\in\pi$ appartengono alla relazione che definisce il problema
		\item esempi: soddisfacibilit\'a, clique, vertex cover, nei quali chiediamo in output un assegnamento di verit\'a soddisfacente, rispettivamente una clique o un vertex cover, invece di semplicemente "si" o "no"
	\end{itemize}
	\item ottimizzazione
	\begin{itemize}
		\item data un'istanza $x\in I_{\pi}$, si chiede di determinare una soluzione $y\in S_{\pi}$ ottimizzando una data misura della funzione costo
		\item esempi: min spanning tree, max SAT, max clique, min vertex cover, min TSP, etc....
	\end{itemize}
\end{itemize}

% $$$$$$$$$$$$$$$$$$$$$$$$$$$$$$$$$$$$$$$$$$$$$$$$$$$$$$$$$$$$$$$$$$$$$$$$$$$$$$$$

\subsection*{complessit\'a degli algoritmi e dei problemi}
\addcontentsline{toc}{subsection}{complessit\'a degli algoritmi e dei problemi}
\begin{itemize}
	\item espressa in funzione della taglia dell'input (denotata come $|x|, \forall x\in I_{\pi}$)
	\item taglia dell'istanza $x$
	\begin{itemize}
		\item quantit\'a di memoria necessaria a memorizzare $x$ in un computer
		\item lunghezza $|x|_{c}$ della stringa che codifica $x$ in un particolare codice naturale $c:I_{\pi}\rightarrow\Sigma$, dove $\Sigma$ \'e l'alfabeto del codice $c$
	\end{itemize}
	\item codice naturale
	\begin{itemize}
		\item conciso: le stringhe che codificano le istanze non devono essere ridondanti o allungate inutilmente
		\item numeri espressi in base $\geq 2$
	\end{itemize}
\end{itemize}

% $$$$$$$$$$$$$$$$$$$$$$$$$$$$$$$$$$$$$$$$$$$$$$$$$$$$$$$$$$$$$$$$$$$$$$$$$$$$$$$$

\subsection*{esempio: codice}
\addcontentsline{toc}{subsection}{esempio: codice}
\begin{flushleft}
	\begin{itemize}
		\item istanza: grafo $G$
	\end{itemize}
\end{flushleft}
\begin{center}
\begin{tikzpicture}
	\Vertex[fontsize=\normalsize, x=0, y=0, label=1, opacity=0]{1}
	\Vertex[fontsize=\normalsize, x=3, y=0, label=2, opacity=0]{2}
	\Vertex[fontsize=\normalsize, x=0, y=-3, label=3, opacity=0]{3}
	\Vertex[fontsize=\normalsize, x=3, y=-3, label=4, opacity=0]{4}
	\Edge[fontsize=\normalsize, position=above, label=2](1)(2)
	\Edge[fontsize=\normalsize, position=left, label=1](1)(3)
	\Edge[fontsize=\normalsize, position=right, label=7](2)(4)
	\Edge[fontsize=\normalsize, position=below, label=4](3)(4)
	\Edge[fontsize=\normalsize, position={above left}, label=3](3)(2)
\end{tikzpicture}
\end{center}
\begin{flushleft}
	\begin{itemize}
		\item codice per $G$
		\begin{itemize}
			\item $\Sigma=\{\{,\},,,0,1,2,3,4,5,6,7,8,9\}$ (simboli)
			\item $c(G)=\{1,2,3,4,\{1,2\},\{1,3\},\{2,3\},\{2,4\},\{3,4\},2,1,3,7,4\}$
			\begin{itemize}
				\item $\{1,2,3,4\}$ (nodi)
				\item $\{\{1,2\},\{1,3\},\{2,3\},\{2,4\},\{3,4\}\}$ (archi)
				\item $\{2,1,3,7,4\}$ (pesi)
			\end{itemize}
			\item $|G|_{c}=49$
		\end{itemize}
	\end{itemize}
\end{flushleft}

% $$$$$$$$$$$$$$$$$$$$$$$$$$$$$$$$$$$$$$$$$$$$$$$$$$$$$$$$$$$$$$$$$$$$$$$$$$$$$$$$

\subsection*{def: tempo di esecuzione dell'algoritmo $A$}
\addcontentsline{toc}{subsection}{def: tempo di esecuzione dell'algoritmo $A$}
\begin{flushleft}
	sia $t_{A}(x)$ il tempo di esecuzione dell'algoritmo $A$ per l'input $x_{i}$, allora il tempo di esecuzione nel caso peggiore di $A$ \'e:
	$$T_{A}(n)=\max\{t_{A}(x)\hspace{0.1cm}|\hspace{0.1cm}|x|\leq n\},\hspace{0.5cm}\forall n>0$$
\end{flushleft}

% $$$$$$$$$$$$$$$$$$$$$$$$$$$$$$$$$$$$$$$$$$$$$$$$$$$$$$$$$$$$$$$$$$$$$$$$$$$$$$$$

\subsection*{def: complessit\'a temporale dell'algoritmo $A$}
\addcontentsline{toc}{subsection}{def: complessit\'a temporale dell'algoritmo $A$}
\begin{flushleft}
	l'algoritmo $A$ ha complessit\'a temporale
	\begin{itemize}
		\item $O(g(n))$ se $T_{A}(n)=O(g(n))$, ovvero $$\lim_{n\to\infty}\frac{T_{A}(n)}{g(n)}\leq c\text{, per una costante }c>0$$
		\item $\Omega(g(n))$ se $T_{A}(n)=\Omega(g(n))$, ovvero $$\lim_{n\to\infty}\frac{T_{A}(n)}{g(n)}\geq c\text{, per una costante }c>0$$
		\item $\Theta(g(n))$ se $T_{A}(n)=\Theta(g(n))$, ovvero $$T_{A}(n)=\Omega(g(n))\text{ e
			}T_{A}(n)=O(g(n))$$
	\end{itemize}
\end{flushleft}

% $$$$$$$$$$$$$$$$$$$$$$$$$$$$$$$$$$$$$$$$$$$$$$$$$$$$$$$$$$$$$$$$$$$$$$$$$$$$$$$$

\subsection*{def: complessit\'a di un problema}
\addcontentsline{toc}{subsection}{def: complessit\'a di un problema}
\begin{flushleft}
	un problema ha complessit\'a
	\begin{itemize}
		\item $O(g(n))$ se esiste un algoritmo che lo risolve avente complessit\'a $O(g(n))$
		\item $\Omega(g(n))$ se ogni algoritmo $A$ che lo risolve ha complessit\'a $\Omega(g(n))$
		\item $\Theta(g(n))$ se ha complessit\'a $O(g(n))$ e $\Omega(g(n))$
	\end{itemize}
\end{flushleft}

% $$$$$$$$$$$$$$$$$$$$$$$$$$$$$$$$$$$$$$$$$$$$$$$$$$$$$$$$$$$$$$$$$$$$$$$$$$$$$$$$

\subsection*{problemi di decisione e classi di complessit\'a}
\addcontentsline{toc}{subsection}{problemi di decisione e classi di complessit\'a}
\begin{flushleft}
	i problemi di decisione sono solitamente descritti da un'istanza di input (o semplicemente INPUT) e da una DOMANDA sull'input
	esempi:
	\begin{itemize}
		\item soddisfacibilit\'a
			\begin{itemize}
				\item INPUT: CNF (Conjunctive Normal Form) formula definita su un insieme di variabili
				\item DOMANDA: esiste un assegnamento di verit\'a $\tau:V\rightarrow\{0,1\}$ ?
			\end{itemize}
		\item clique
			\begin{itemize}
				\item INPUT: un grafo non orientato $G=(V,E)$ di $n$ nodi e un intero $k>0$
				\item DOMANDA: esiste in $G$ una clique di almeno $k$ nodi ($>k$), ovvero un sottoinsieme $U\subseteq V$ tale che $|U|\geq K$ e $\{u,v\}\in E$, $V\in U$ ?
			\end{itemize}
		\item vertex cover
			\begin{itemize}
				\item INPUT: un grafo non orientato $G=(V,E)$ di $n$ nodi e un intero $k>0$
				\item DOMANDA: esiste in $G$ una vertex cover di al massimo $k$ nodi ($<k$), ovvero un sottoinsieme $U\subseteq V$ tale che $|U|\leq K$ e $u\in U$ o $v\in U$, $\forall\{u,v\}\in E$ ?
			\end{itemize}
	\end{itemize}
	nei problemi di decisione $I_{\pi}=Y_{\pi}\cup N_{\pi}$
	\begin{itemize}
		\item $Y_{\pi}=$ insieme di istanze positive, ovvero con soluzione $1$
		\item $N_{\pi}=$ insieme di istanze negative, ovvero con soluzione $0$
	\end{itemize}
\end{flushleft}

% $$$$$$$$$$$$$$$$$$$$$$$$$$$$$$$$$$$$$$$$$$$$$$$$$$$$$$$$$$$$$$$$$$$$$$$$$$$$$$$$

\subsection*{def: un algoritmo $A$ risolve $\pi$}
\addcontentsline{toc}{subsection}{def: un algoritmo $A$ risolve $\pi$}
\begin{flushleft}
	un algoritmo $A$ risolve $\pi\iff\forall$ input $x\in I_{\pi}$, $A$ risponde $1\iff x\in Y_{\pi}$
\end{flushleft}

% $$$$$$$$$$$$$$$$$$$$$$$$$$$$$$$$$$$$$$$$$$$$$$$$$$$$$$$$$$$$$$$$$$$$$$$$$$$$$$$$

\subsection*{def: classe dei problemi $TIME(g(n))$}
\addcontentsline{toc}{subsection}{def: classe dei problemi $TIME(g(n))$}
\begin{flushleft}
	$TIME(g(n))=$ classe dei problemi di decisione con complessit\'a $O(g(n))$
\end{flushleft}

% $$$$$$$$$$$$$$$$$$$$$$$$$$$$$$$$$$$$$$$$$$$$$$$$$$$$$$$$$$$$$$$$$$$$$$$$$$$$$$$$

\newpage
